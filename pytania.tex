\documentclass[main.tex]{subfiles}
\graphicspath{
    {"../img/"}
    {"img/"}
}

\begin{document}
    \begin{enumerate}
        \item
            \begin{enumerate}
                \item Iloczyn wewnętrzny
                \item Zbiory ściągalne
                \item Lemat Poincare
                \item przykłady form zamkniętych a niezupełnych
            \end{enumerate}
        \item
            \begin{enumerate}
                \item Orientacja
                \item Całkowanie form różniczkowych
                \item Twierdzenie Stokesa
                \item Całkowa postać równań Maxwella
            \end{enumerate}
        \item Objętość rozmaitości (wzór na długość łuku, pole powierzchni, itp.)
        \item
            \begin{enumerate}
                \item Funkcje holomorficzne
                \item Równania Cauchy-Riemann
                \item Różniczkowalność w sensie zespolonym
            \end{enumerate}
        \item
            \begin{enumerate}
                \item Twierdzenie Cauchy
                \item Wzór Cauchy
                \item Twierdzenie Liouville
                \item Zasadnicze Twierdzenie Algebry v1.0
            \end{enumerate}
        \item
            \begin{enumerate}
                \item Zera funkcji holomorficznej
                \item Rozwinięcie funkcji holomorficznej w szereg potęgowy
                \item Przedłużenie analityczne
            \end{enumerate}
        \item
            \begin{enumerate}
                \item Funkcje holomorficzne w pierścieniu
                \item Szereg Laurent
                \item Przedłużenie analityczne
            \end{enumerate}
        \item
            \begin{enumerate}
                \item Klasyfikacja punktów izolowanych
                \item Twierdzenie o residuach
            \end{enumerate}
        \item
            \begin{enumerate}
                \item Lemat Jordan
                \item Punkt w nieskończoności
                \item Jednoznaczność funkcji zespolonych
                \item Przedłużenie analityczne
            \end{enumerate}
        \item
            \begin{enumerate}
                \item Twierdzenie Weierstrass
                \item Twierdzenie Rouche i konsekwencje
                \item Zasadnicze twierdzenie algebry v2.0
            \end{enumerate}
        \item
            \begin{enumerate}
                \item Wzór na sumowanie szeregów potęgowych
                \item Przekształcenie konforemne
                \item Krzywizna
                \item Przykład zastosowania twierdzenia Kasnera-Arnolda
            \end{enumerate}
        \item
            \begin{enumerate}
                \item Transformata Fouriera funkcji całkowalnych
                \item Własności
                \item Transformata odwrotna
                \item Splot
            \end{enumerate}
        \item Równanie przewodnictwa
        \item
            \begin{enumerate}
                \item Wzór Plancherela
                \item Nierówność Heisenberga
            \end{enumerate}
        \item
            \begin{enumerate}
                \item Dystrybucje
                \item Definicje
                \item Podstawowe własności
                \item Przykłady
                \item Równanie dystrybucyjne $x T = 0$
            \end{enumerate}
        \item Wzór Greena $\Delta \frac{1}{r}$
        \item
            \begin{enumerate}
                \item Równanie dystrybucyjne $x T = 0$
                \item Dystrybucje temperowane
                \item Transformata Fouriera dystrybucji
                \item podstawowe własności i przykłady $(\hat{1}, \hat{\delta})$
            \end{enumerate}
        \item Twierdzenie o próbkowaniu Shannona
        \item
            \begin{enumerate}
                \item Równanie dystrybucyjne $x T = 0$
                \item Dystrybucje temperowane
                \item Wzór sumacyjny Poisson
            \end{enumerate}
        \item Zbieżność szeregów Fouriera
    \end{enumerate}
\end{document}
