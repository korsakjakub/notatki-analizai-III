\documentclass[../main.tex]{subfiles}
\graphicspath{
    {"../img/"}
    {"img/"}
}

\begin{document}
    \subsection{Refleksja}
    Czy to
    \begin{align*}
        \frac{\partial P}{\partial x} &= \frac{\partial Q}{\partial y} \\
        \frac{\partial P}{\partial y} &= - \frac{\partial Q}{\partial x}
    \end{align*}
    jest fajne?
    \begin{przyklad}
        \[
            \nabla P = \left[ \frac{\partial P}{\partial x} , \frac{\partial P}{\partial y}  \right]
        ,\]
    \[
        \nabla Q = \left[ \frac{\partial Q}{\partial x} , \frac{\partial Q}{\partial y}  \right]
    ,\]
to możemy zrobić takie coś:
\[
    "\left( \nabla P \cdot \nabla Q \right) " = \frac{\partial P}{\partial x} \frac{\partial Q}{\partial x} + \frac{\partial P}{\partial y} \frac{\partial Q}{\partial y} = - \frac{\partial P}{\partial x} \frac{\partial P}{\partial y} + \frac{\partial P}{\partial y} \frac{\partial P}{\partial x} = 0
.\]
    \end{przyklad}
    \begin{tw}
        $f$ - holomorficzna na $\mathcal{O}\subset\mathbb{C}$, $\mathcal{O}$ - otwarty wtedy i tylko wtedy, gdy $f$ - spełnia warunek Cauchy-Riemanna.
    \end{tw}
    \begin{proof}
        $\implies$ było\\
        $\impliedby$ Zauważmy, że skoro $P(x,y)$, $Q(x,y)$ spełniają warunki Cauchy-Riemanna, to znaczy, że funkcja
        \[
            F(x,y) = \begin{bmatrix} P(x,y)\\Q(x,y) \end{bmatrix}
        ,\]
    $F: U\subset\mathbb{R}^2\to \mathbb{R}^2$ jest różniczkowalna na $U\subset\mathbb{R}^2$, czyli dla $h = \begin{bmatrix} h_1\\h_2 \end{bmatrix} $ jest
        \[
            \underbrace{F(x+h_1, y+h_2) - F(x,y)}_{\Delta F} = \begin{bmatrix} \frac{\partial P}{\partial x} & \frac{\partial P}{\partial y} \\ \frac{\partial Q}{\partial x} & \frac{\partial Q}{\partial y}  \end{bmatrix} \begin{bmatrix} h_1\\h_2 \end{bmatrix} + r(x,y,h)
        ,\]
    $\frac{r(x,y,h)}{\left\Vert h \right\Vert } \underset{h \to 0}{\longrightarrow} 0$.\\
        Czyli
        \[
            \underbrace{\overbrace{\begin{bmatrix} P(x+h_1, y+h_2) - P(x,y)\\ Q(x+h_1, y+h_2) - Q(x,y) \end{bmatrix}}^{\Delta P}}_{\Delta Q} \overset{\text{C-R}}{=} \begin{bmatrix} \frac{\partial P}{\partial x} & - \frac{\partial Q}{\partial x} \\ \frac{\partial Q}{\partial x} & \frac{\partial P}{\partial x}  \end{bmatrix} \begin{bmatrix} h_1\\h_2 \end{bmatrix} + r(x,y,h)
        ,\]
    zatem
        \[
            \begin{bmatrix} \Delta P\\ \Delta Q \end{bmatrix} = \begin{bmatrix} a& -b\\ b & a \end{bmatrix} \begin{bmatrix} h_1\\h_2 \end{bmatrix} + r(x,y,h)
        .\]
     to wygląda trochę jak obrót. Dalej
        \[
            \begin{bmatrix} \Delta P\\ \Delta Q \end{bmatrix} = \begin{bmatrix} a h_1 - b h_2\\ b h_1 + a h_2 \end{bmatrix} + r(x,y,h)
        .\]
    Ale
    \begin{align*}
        f(z+h) - f(z) &= P(x+h_1, y+h_2) + i Q(x+h_1, y+h_2) - \left( P(x,y) + i Q(x,y) \right) =\\
        &= \Delta P + i \Delta Q = a h_1 - bh_2 + i \left( bh_1 + ah_2 \right) + r = \\
        &= \left( a + ib \right) \left( h_1 + ih_2 \right) + r
    ,\end{align*}
    zatem
        \[
            \frac{f(z+h) - f(z)}{h} = a + ib + \frac{r}{h}
        .\]
    A jak przejdzie się z $h$ do $0$, to $\frac{r}{h} \to 0$, więc
        \[
            \lim\limits_{h \to 0} \frac{f(z+h) - f(z)}{h} = f'(z)
        \]
    \end{proof}
    \begin{stw}
        Niech $f: \mathcal{O}\subset\mathbb{C}\to U\subset\mathbb{C}$, $f$ - holomorficzna na $\mathcal{O}$, a $g: U \to \mathbb{C}$ - holomorficzna na $U$. Wówczas $g\circ f$ - holomorficzna na $\mathcal{O}$.
    \end{stw}
    \begin{proof}
        \begin{align*}
            \left( g\circ f \right) ' &= g'(f) f' = \begin{bmatrix} a& -b\\ b&a \end{bmatrix} \cdot \begin{bmatrix} a_1& -b_1\\ b_1& a_1 \end{bmatrix} =\\
                &= \begin{bmatrix} a a_1 - b b_1 & -ab_1 -a_1b\\ a_1b + ab_1 & -bb_1 + aa_1\end{bmatrix} =\\
                    &= \begin{bmatrix} aa_1 - bb_1& -\left( a_1b + ab_1 \right) \\ a_1b + ab_1 & aa_1 - bb_1 \end{bmatrix}
        ,\end{align*}
    a tak wygląda macierz pochodnej $f$ - holomorficznej (traktowanej jako funkcja z $\mathbb{R}^2\to \mathbb{R}^2$).
    \end{proof}
    \subsection{Oznaczenia}
    niech $M\subset\mathbb{R}^2$, $\left<dx, dy \right> = T_p^* M$. Wprowadźmy
    \begin{align*}
        dz &= dx + idy\\
        d\bar{z} &= dx - idy
    .\end{align*}
    Jeżeli $f(x,y): \mathbb{R}^2\to \mathbb{R}^1$, to
    \begin{align*}
        df &= \frac{\partial f}{\partial x} dx + \frac{\partial f}{\partial y} dy = \frac{1}{2} \frac{\partial f}{\partial x} \left( dz + d\bar{z} \right) + \frac{1}{2i} \frac{\partial f}{\partial y} \left( dz - d\bar{z} \right) =\\
        &= \left( \frac{1}{2}\frac{\partial f}{\partial x} + \frac{1}{2i}\frac{\partial f}{\partial y}  \right) dz + \underbrace{\left( \frac{1}{2}\frac{\partial f}{\partial x} - \frac{1}{2i}\frac{\partial f}{\partial y}  \right)}_{\frac{\partial f}{\partial \bar{z}} }d\bar{z}
    .\end{align*}
\textbf{Obserwacja:} niech $f(z) = P(x,y) + i Q(x,y)$, wówczas
\begin{align*}
    \frac{\partial f}{\partial x} = \frac{\partial P}{\partial x} + i \frac{\partial Q}{\partial x} \\ \frac{\partial f}{\partial y} = \frac{\partial P}{\partial y} + i \frac{\partial Q}{\partial y}
.\end{align*}
czyli
\begin{align*}
    \frac{\partial f}{\partial \bar{z}} &= \frac{1}{2} \left( \frac{\partial f}{\partial x} - \frac{1}{i} \frac{\partial f}{\partial y}  \right) = \\
    &= \frac{1}{2} \left( \frac{\partial P}{\partial x} + i \frac{\partial Q}{\partial x} - \frac{1}{i}\left( \frac{\partial P}{\partial y} + i \frac{\partial Q}{\partial y}  \right)  \right)  =\\
    &= \frac{1}{2} \left(\left( \frac{\partial P}{\partial x}  - \frac{\partial Q}{\partial y}  \right) + i \left(\frac{\partial Q}{\partial x} + \frac{\partial P}{\partial y}\right)\right)  \\
.\end{align*}
\begin{przyklad}
    $f(z) = z^2 = z\cdot z$,\\
    \[
        \frac{\partial f}{\partial z} = 2z,\quad \frac{\partial f}{\partial \bar{z}} = 0
    \]
    a $g(z) = |z|^2 = z\cdot \bar{z}$
    \[
        \frac{\partial g}{\partial \bar{z}} = z \neq 0
    .\] Czyli $g$ - nie jest holomorficzna
\end{przyklad}
\begin{przyklad}
    Obliczmy całkę:
    \[
        \int_{\partial K(0,r)} \frac{dz}{z} = \begin{vmatrix} z = re^{i\theta}\\ dz = rie^{i\theta}d\theta \end{vmatrix} = \int_0^{2\pi} \frac{rie^{i\theta}d\theta}{re^{i\theta}} = i \int_0^{2\pi}d\theta = 2\pi i
    .\]
\end{przyklad}
\begin{stw}
    Jeżeli $f$ - holomorficzna na $\mathcal{O}$ i $\Omega \subset\mathcal{O}$, to
    \[
        \int_{\partial \Omega} fdz = \int_\Omega d(fdz) = \int_{\Omega} \frac{\partial f}{\partial d\bar{z}} d\bar{z}\land dz = 0
    .\]
\end{stw}
\begin{tw}
    (wzór Cauchy)\\
    Niech $\Omega\subset\mathbb{C}$, $f: \bar{\Omega} \to \mathbb{C}$, niech $\xi \in \Omega$. Wówczas  \[
        f(\xi) = \frac{1}{2\pi i }\int_{\partial \Omega}\frac{f(z)}{z-\xi}dz + \int_{\Omega} \frac{1}{z - \xi}\frac{\partial f}{\partial \bar{z}} dz\land d\bar{z}
    .\]
\end{tw}
\textbf{Obserwacja:} jeżeli $f$ - holomorficzna na $\Omega$, to
\[
    f(\xi) = \frac{1}{2\pi i} \int_{\partial \Omega}\frac{f(z)}{z - \xi} dz
.\]
Wynik $\frac{1}{2\pi i }\int_{\partial K(0,r)} \frac{dz}{z} = 1$ otrzymamy dla $\xi = 0$ i $f(z) = 1$
 \begin{proof}
     niech
     \[
         g(z) = \frac{f(z)}{z - \xi}
     .\]
     zatem wiemy, że
     \[
         \int_{\partial \Omega_\epsilon}g(z) = \int_{\Omega} d g(z)
     .\]
 \[
     \int_{\partial \Omega} \frac{f(z)}{z - \xi}dz + \int_{\partial K(\xi, \epsilon)} \frac{f(z)}{z - \xi}dz = \int\int\limits_{\Omega_\epsilon}\frac{1}{z-\xi} \frac{\partial f}{\partial \bar{z}} d\bar{z}\land dz
 .\]
 \textbf{Pytanie:} co się dzieje, jak przejdziemy z $\epsilon\to 0$ Oznacza to, że chcemy zbadać zachowanie takiej całki
     \[
         \int\int\limits_{\Omega_{\epsilon}} \frac{1}{z-\xi}\frac{\partial f}{\partial \bar{z}}
     \]
 dla $z = \epsilon e^{i\theta} + \xi$, ale
     \[
         \frac{1}{\epsilon e^{i\theta} + \xi - \xi} = \frac{e^{-i\theta}}{\epsilon}
     ,\]
 a całka $\int\int\limits_{\Omega_\epsilon} d\bar{z}\land dz \approx \underbrace{\epsilon d\epsilon d\theta}_{\text{element powierzchni}}$.
     Oznacza, to że
     \[
         \frac{1}{z-\xi} d\bar{z}\land dz \overset{\epsilon \to 0}{\approx} \frac{1}{\epsilon} \cdot \epsilon
     ,\]
 czyli w $\epsilon = 0$ nie wybuchnie!\\
     Ale
     \[
         \int_{\partial K(\xi, \epsilon)} \frac{f(z)}{z-\xi} dz = - \int_0^{2\pi} \frac{f(\xi + \epsilon e^{i\theta})}{\epsilon e^{i\theta}} \epsilon i e^{i\theta} d\theta =
     .\]
 Trzeba wrzucić twierdzenie o wartości średniej
     \[
         = i f(c) \cdot \int_{0}^{2\pi}d\theta = 2 \pi i f(c) \underset{\epsilon \to 0}{\longrightarrow}-2\pi i f(\xi)
     ,\]
 gdzie $c\in \partial K(\xi, \epsilon)$.\\
     Zatem
     \[
         \int_{\partial \Omega} \frac{f(z)}{z - \xi}dz - \int_{\Omega}\frac{1}{z-\xi}\frac{\partial f}{\partial \bar{z}} d\bar{z}\land dz = 2 \pi i f(\xi)
     .\]
\end{proof}
\begin{tw}
    (Liouville)\\
    Jeżeli $f$ - ograniczona i holomorficzna na całym $\mathbb{C}$, to $f$ jest stała.
\end{tw}
\textbf{Obserwacja:} a co z sinusem?
$f(x) = \sin(x)$, ale trzeba zastanowić się nad  $f(z) = \sin(z) = \frac{e^{iz} - e^{-iz}}{2i}$. Dla np. $z = it$,
\[
    \sin(it) = \frac{e^{-t} - e^{t}}{2i}
,\]
czyli oczywiście sinus ograniczony nie jest.

\end{document}
