\documentclass[../main.tex]{subfiles}
\graphicspath{
    {"../img/"}
    {"img/"}
}

\begin{document}
    \begin{tw}
        (Liouville)\\
        Jeżeli $f$ - holomorficzna i ograniczona na $\mathbb{C}$, to $f$ - stała.
    \end{tw}
    \begin{proof}
        Wiemy, że
        \[
            \underset{M > 0}{\exists} \quad \underset{z\in\mathbb{C}}{\forall} \quad |f(z)| < M
        .\]
    Skoro $f$ - holomorficzna, to znaczy, że dla $\xi\in\mathbb{C}$,
        \[
            f(\xi) = \frac{1}{2\pi i}\int\limits_{\partial K(\xi, r)}\frac{f(z)}{z-\xi}dz
        .\]
    (Wzór Cauchy)\\
    Zauważmy, że skoro $f$ - jak wyżej, to
        \[
            f'(\xi) = \frac{1}{2\pi i} \int\limits_{\partial K(\xi,r)} \frac{f(z)}{(z - \xi)^2}dz
        .\]
    (Absolutnie nieoczywiste lol. Uzasadnienie później)\\
        Wówczas możemy oszacować $f'$
        \begin{align*}
            |f'(\xi)| &\le \left|\frac{1}{2\pi i}\right| \max\limits_{z\in \partial K(\xi,r)}\left| \frac{f(z)}{(z-\xi)^2}\right|\cdot \left| \text{długość okręgu } K(\xi,r) \right| =\\
            &= \frac{1}{2\pi}\cdot \frac{M}{\left| \left(\xi + re^{i\varphi} - \xi\right)^2 \right| } \left| 2\pi r \right| = \frac{1}{2\pi} \frac{M}{r^2} 2\pi r = \frac{M}{r} \quad\underset{r>0}{\forall}
        .\end{align*}
    Czyli
        \[
            \underset{r>0}{\forall} \left| f'(\xi) \right| < \frac{M}{r} \underset{r\to\infty}{\longrightarrow} 0
        .\]
    Zatem $\left| f'(\xi) \right| = 0$,
        czyli
        \[
            f(z) = const
        .\]
    \end{proof}
    \begin{przyklad}
        $f(z) = \sin(z) = \frac{e^{iz} - e^{-iz}}{2i}$ jest holomorficzna na $\mathbb{C}$, ale nie jest na $\mathbb{C}$ ograniczona (tylko dla $z\in\mathbb{R}$ ).
    \end{przyklad}
    \textbf{Wniosek:} (Zasadnicze Twierdzenie Algebry)\\
    Niech $w(z) = a_nz^n + a_{n-1}z^{n-1} + \ldots + a_0$.\\
    Załóżmy, że
    \[
        \underset{z\in\mathbb{C}}{\forall} \quad w(z) \neq 0
    .\]
Oznacza to, że
\[
    f(z) = \frac{1}{w(z)}\text{ jest na } \mathbb{C} \text{ holomorficzna i ograniczona}
.\]
Jest więc stała. Co oznacza, że $w(z)$ jest stała i sprzeczność. $\Box$\\
(PS oznacza to, że $\underset{z_0\in\mathbb{C}}{\exists} $, że $w(z_0) = 0$, czyli $w(z) = (z-z_0)w_1(z)$. Biorąc funkcję $f_1(z) = w_1(z)\ldots$ pokażemy, że wielomian stopnia $n$ nad $\mathbb{C}$ ma $n$ pierwiastków.$\quad\Box$)
\subsection{Szeregi Laurenta}
\begin{przyklad}
    Niech
    \[
        f(z) = \frac{z+1}{z^2 + 1}
    .\]
Zauważmy, że
\[
    f(z) = \frac{z+1}{z^2 + 1} = \frac{1}{2}\frac{1-i}{z - i} + \frac{1}{2}\frac{1+i}{z+i}
.\]
Jeżeli
\[
\left| z+2i \right| < 3
,\]
to
\begin{align*}
    \frac{1}{z-i} &= \frac{1}{z + 2i - 3i} = \frac{1}{-3i}\cdot \frac{1}{1-\frac{z+2i}{3i}} =\\
    &= -\frac{1}{3i}\sum_{n=0}^{\infty} \left( \frac{z+2i}{3i} \right) ^{n} = - \sum_{n=0}^{\infty} \frac{1}{(3i)^{n+1}}\left( z+2i \right)^n
.\end{align*}
Jeżeli $|z+2i| > 1$, to
\begin{align*}
    \frac{1}{z+1} &= \frac{1}{z + 2i - i} = \frac{1}{z+2i}\cdot \frac{1}{1- \frac{i}{z+2i}} =\\
    &= \frac{1}{z+2i} \cdot \sum_{n=0}^{\infty} \left( \frac{i}{z+2i} \right) ^n = \sum_{n=0}^{\infty} \frac{(i)^n}{(z+2i)^{n+1}} =\\
    &=\sum_{n=1}^{\infty} \frac{(i)^{n-1}}{(z+2i)^n}
.\end{align*}
Zatem
\[
    \frac{z+1}{z^2 + 1} = \frac{1+i}{2} \cdot \sum_{n=1}^{\infty} \frac{(i)^{n-1}}{(z+2i)^n} + \frac{i-1}{2} \sum_{n=0}^{\infty} \frac{1}{(3i)^n}(z+2i)^n = \sum_{n=-\infty}^{\infty} d_k (z+2i)^k
,\]
gdzie
\[
dk = \begin{cases}
    \frac{1+i}{2}\cdot (i)^{-k-1} & k <0 \\
    \frac{i - 1}{2}\cdot \frac{1}{(3i)^{k}} & k \ge 0
\end{cases}
.\]
Niech
    \[
        R(2i, 1, 3) \overset{\text{def}}{=} \left\{ z\in\mathbb{C}, |z+2i| < 3 \land |z + 2i| > 1 \right\}
    \]
    - pierścień otwarty o środku $2i$ i promieniach $1$ i $3$.\\
    Dla $|z+2i| < 1$,
    \begin{align*}
        \frac{1}{z+i} &= \frac{1}{z+2i-i} = -\frac{1}{i} \cdot \frac{1}{1 - \frac{z+2i}{i}} =\\
        &= -\frac{1}{i} \cdot \sum_{n=0}^{\infty} \left(\frac{z+2i}{i}\right)^n = -\sum_{n=0}^{\infty} \frac{1}{(i)^{n+1}}\frac{(z+2i)^n}{1}
    .\end{align*}
Zatem dla $z\in R(-2i, 0, 1)$,
\begin{align*}
    f(z) &= \frac{z+1}{z^2 + 1} = \frac{1+i}{2}\cdot \sum_{n=0}^{\infty} \frac{1}{(i)^{n+1}}(z+2i)^n -\\
    &- \frac{1-i}{2}\cdot \sum_{n=0}^{\infty} \frac{1}{(3i)^{n+1}}\cdot (z+2i)^n = \sum_{k = 0}^{\infty}d_k \left( z+2i \right) ^k
,\end{align*}
gdzie
\[
    d_k = -\frac{1+i}{2}\cdot \frac{1}{(i)^{n+1}} - \frac{1-i}{2}\cdot \frac{1}{(3i)^{n+1}}
.\]
dla $|z+2i| > 3$
\begin{align*}
    \frac{1}{z-i} &= \frac{1}{z+2i-3i} = \frac{1}{z+2i}\cdot \frac{1}{1 - \frac{3i}{z+2i}} =\\
    &= \frac{1}{z+2i}\cdot \sum_{n=0}^{\infty} (3i)^n\cdot \frac{1}{(z+2i)^n} =\\
    &= \sum_{n=0}^{\infty} \frac{(3i)^n}{(z+2i)^{n+1}} = \sum_{n=1}^{\infty} \frac{(3i)^{n-1}}{(z+2i)^n}
.\end{align*}
I wtedy dla $z\in R(-2i,3,+\infty)$, jest
\[
    \frac{z+1}{z^2+1} = \frac{1+i}{2} \cdot \sum_{n=1}^{\infty} \frac{(i)^n}{(z+2i)^n} + \frac{1-i}{2}\cdot \sum_{n=1}^{\infty} \frac{(3i)^{n-1}}{(z+2i)^n} = \sum_{k=-1}^{-\infty} d_k (z+2i)^k
.\]
\end{przyklad}
\begin{tw}
    (Laurent)\\
    Niech $f(z)$ - holomorficzna na pierścieniu $R(z_0,r_1,r_2)$,
    \[
        R(z_0,r_1,r_2) := \left\{ z\in\mathbb{C}, |z-z_0|>r_1 \land |z-z_0| < r_2 \right\}
    .\]
Wówczas $\underset{z\in R(z_0,r_1,r_2)}{\forall} $
\[
    f(z) = \sum_{n=-\infty}^{+\infty} a_n(z-z_0)^n
,\]
gdzie
\[
    a_n = \frac{1}{2\pi i} \oint\limits_{\partial K(z_0,r)} \frac{f(\xi)}{(\xi - z_0)^{n+1}}
,\]
$r_1<r<r_2$
\end{tw}
\begin{proof}
    Zauważmy, że
        \[
            \underset{z\in R(z_0,r_1,r_2)}{\forall}\quad \underset{r_1' > r_1}{\exists}\quad \underset{r_2' < r_2}{\exists} z\in R(z_0,r_1',r_2').
        \]
        Ze wzoru Cauchy wiemy, że
        \[
            f(z) = \frac{1}{2\pi i } \int\limits_{\partial R(z_0,r_1',r_2')}\frac{f(\xi)}{\xi-z}d\xi = \frac{1}{2\pi i}\left[\int\limits_{\partial K(z_0,r_2')}\frac{f(\xi)}{\xi - z}d\xi \quad- \int\limits_{\partial K(z_0,r_1')} \frac{f(\xi)}{\xi - z}d\xi\right]
        .\]
    ale
    \[
    \frac{1}{\xi - z} = \frac{1}{\xi - z_0 + z_0 - z}
    ,\]
a dla $\xi \in \partial K (z_0,r_1')$ i $z\in K(z_0,r_1')$
\[
    \left|\frac{z-z_0}{\xi-z_0}\right| < 1
.\]
więc
\begin{align*}
    \frac{1}{\xi - z_0 + z_0 - z} &= \frac{1}{\xi - z_0}\cdot \frac{1}{1 + \frac{z_0 - z}{\xi - z_0}} = \frac{1}{\xi - z_0}\cdot \frac{1}{1 - \frac{z-z_0}{\xi - z_0}} =\\
    &=\frac{1}{\xi - z_0}\cdot \sum_{n=0}^{\infty} \left(\frac{z-z_0}{\xi - z_0}\right)^n = \sum_{n=0}^{\infty} \frac{1}{(\xi - z_0)^{n+1}}\cdot (z-z_0)^n
.\end{align*}
więc
\[
    \frac{1}{2 \pi i}\int\limits_{\partial K(z_0,r_1')}\frac{f(\xi)}{\xi - z}d\xi = \frac{1}{2\pi i} \sum_{n=0}^{\infty}\quad \int\limits_{\partial K(z_0,r_1')}\frac{f(\xi)}{(\xi - z_0)^{n+1}}d\xi (z-z_0)^n
    .\]
A dla $\xi\in\partial K(z_0,r_2')$ i $z$ takich, że $|z - z_0| > r_2'$, wiemy, że
\[
    \left|\frac{\xi - z_0}{z - z_0}\right| < 1
\]
itd.
\end{proof}
\end{document}
