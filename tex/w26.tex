\documentclass[../main.tex]{subfiles}
\graphicspath{
    {"../img/"}
    {"img/"}
}

\begin{document}
Z ostatniego odcinka wiemy, że
\[
    \Delta\left( \frac{1}{r} \right) = - 4\pi \delta
.\]
\[
    \left< \Delta \left( \frac{1}{r} \right) , \varphi \right> = -4 \pi \left<\delta, \varphi \right>
.\]
\[
    \Delta \frac{1}{\left| \overline{r} - \overline{r_0} \right| } = -4 \pi \delta (\overline{r} - \overline{r_0})
.\]
Były też kiedyś równania Maxwella
\begin{align*}
    div(E) &= \rho(\overline{r})\\
    E &= - grad(\varphi)\\
    \varphi : \mathbb{R}^3 &\to \mathbb{R}\\
    rot(E) &= -\frac{\partial B}{\partial t}\\
    \frac{\partial B}{\partial t} &= 0
.\end{align*}
Jak to złożymy, to będzie
\[
    \Delta \varphi = -\rho(x)
.\]
\[
    \int\limits_{V} \left( U\Delta V - V \Delta U \right) dV = \int\limits_{\partial V}\left( U \frac{\partial V}{\partial n} - V \frac{\partial U}{\partial n}  \right)dS
.\]
\[
    V = \frac{1}{\left| \overline{r} - \overline{r_0} \right| }
.\]
\[
    U = \varphi(\overline{r})
.\]
Czyli
\[
    \int\limits_{V}\varphi(\overline{r}) \Delta \frac{1}{\left| \overline{r} - \overline{r_0} \right| }d\overline{r} - \int\limits_{V}\frac{1}{\left| \overline{r} - \overline{r_0} \right| }\Delta \varphi d\overline{r} = \int\int\limits_{\partial V}\left(\varphi(\overline{r})\frac{\partial }{\partial n} \frac{1}{\left| \overline{r} - \overline{r_0} \right|} - \frac{1}{\left| r - r_0 \right|}\frac{\partial \varphi}{\partial n} \right) dS
.\]
\[
    -4\pi \int\limits_V \varphi(r)\delta(\overline{r} - \overline{r_0})d\overline{r} - \int\limits_V \frac{-\rho(\overline{r})}{\left| \overline{r} - \overline{r_0} \right| }d\overline{r} = \int\limits_{\partial V}\left(\varphi\left( \overline{r} \right) \frac{\partial }{\partial n} \left( \frac{1}{\left| \overline{r} - \overline{r_0} \right| } \right) - \frac{1}{\left| r-r_0 \right| \frac{\partial \varphi}{\partial n} }\right) dS
.\]

\[
    \varphi(r_0) = \frac{1}{4\pi}\int\limits_V \frac{\rho(\overline{r})}{\left| \overline{r} - \overline{r_0} \right| }d\overline{r} + \frac{1}{4\pi} \int\limits_{\partial V}\frac{1}{\left| \overline{r} - \overline{r_0} \right|} \frac{\partial \varphi}{\partial n} - \varphi(r)\frac{\partial }{\partial n} \left( \frac{1}{\left| \overline{r} - \overline{r_0} \right| } \right)dS
.\]
Druga całka znika czasami w $V\to \infty$ i wtedy zostaje Prawo Coulomba.
\subsection{Równanie $xT = 0$}
 \[
    x T = 0,\quad T\in D^\star
.\]
To znaczy, że
\[
    \underset{\varphi\in D}{\forall} \left<xT, \varphi \right> = 0
.\]
Zauważmy, że
\[
    \left<xT, \varphi \right> = \left<T, x\varphi \right> = 0
.\]
Oznacza to, że dystrybucja $T$ zeruje się na wszystkich funkcjach postaci $x\varphi$, $\varphi\in D$.
\begin{pytanie}
    Czy oznacza to, że $T$ zeruje się na każdej funkcji, która w $x = 0$ wynosi zero?
\end{pytanie}
Załóżmy, że $T$ istnieje i
\[
    \underset{\psi\in D}{\exists} \left<T,\psi \right> = 0
.\]
Oznacza to, że
\[
    \left< xT, \frac{\psi}{x} \right> = 0
.\]
Czyli jeżeli $\psi\in D$, to $\frac{\psi}{x}$ też musi należeć do $D$.
\begin{pytanie}
    Ile wynosi $\psi(0)$?
\end{pytanie}
Gdyby $\psi(0) \neq 0$, to wtedy $\frac{\psi(x)}{x}$ nie byłoby ograniczone w $x = 0$, czyli  $\frac{\psi}{x}\not\in D$
Zauważmy, że
\[
    \frac{\psi(x)}{x} = \int\limits_{0}^{1} \psi'(xt)dt
.\]
Czyli jeżeli $\psi\in D$, to znaczy, że $\psi'\in D$.
Niech  $\varphi(x)$ - dowolne $\in D$ i niech $\alpha(x)$ takie, że $\alpha(0) = 1$, $\alpha\in D$. Wówczas
\[
    \varphi(x) = \varphi(x) - \alpha(x)\varphi(0) + \alpha(x)\varphi(0)
.\]
Wówczas
\[
    \left<T, \varphi \right> = \left<T, \varphi(x) - \alpha(x)\varphi(0) \right> + \left<T, \alpha(x)\varphi(0) \right>
.\]
to pierwsze daje zero, bo liczymy $T$ na funkcji, która  w zerze daje zero.
Zatem
\[
    \underset{\varphi\in D}{\forall}\quad \left<T,\varphi \right> = \left<T, \alpha(x) \right>\varphi(0)
.\]
Czyli $\left<T,\varphi \right> = C_\alpha \varphi(0) = \left<C_\alpha \delta, \varphi \right>$, czyli $T = C_\alpha \delta$.
\begin{pytanie}
    Czy $C_\alpha$ rzeczywiście zależy od wyboru funkcji $\alpha(x)$, czy jest stałą uniwersalną?
\end{pytanie}
\subsection{Transformata Fouriera dystrybucji}
\[
    \left<\mathcal{F}T, \varphi \right> \overset{\text{def}}{=} \left<T, \mathcal{F}\varphi \right>
.\]
\[
    T\in S^\star, \underset{\varphi\in S}{\forall}
.\]
\begin{definicja}
    (Przestrzeń Schwartza)\\
    Przestrzenią Schwartza ($S$) nazywamy zbiór takich $\varphi\in C^\infty(\mathbb{R})$, że
    \begin{enumerate}
        \item $\underset{L, m \ge 0}{\forall} x^L\varphi^{(m)}$ - ograniczone (w sensie $\left\Vert . \right\Vert $ )
        \item $\underset{L, m \ge 0}{\forall} \left( x^L \varphi \right)^{(m)} $ jest całkowalna
    \end{enumerate}
\end{definicja}
\textbf{Motywacja:}
\[
    \mathcal{F}\left( \varphi' \right)  \sim x \mathcal{F}\varphi
\]
\[
    \mathcal{F}'\left( x\varphi \right) \sim \mathcal{F}'(\varphi)
.\]
\begin{definicja}
    Przestrzeń dualną do $S$ oznaczamy, przez $S^\star$, odwzorowania liniowe z $S^\star$ nazywamy dystrybucjami temperowanymi.
\end{definicja}
Policzmy nareszcie $\mathcal{F}\delta$
\[
    \left<\mathcal{F}\delta, \varphi \right> = \left<\delta, \mathcal{F}\varphi \right> = \left( \mathcal{F}\varphi \right) (0) = \int\limits_{-\infty}^{\infty} e^{-2\pi i k \cdot 0}\varphi(k)dk = \int\limits_{-\infty}^{\infty} 1\cdot \varphi(k)dk = \left<1, \varphi \right>
.\]
Zatem $\mathcal{F}\delta = 1$. A ile wynosi $\mathcal{F}\delta(x-a)$?
\[
    \left<\mathcal{F}\delta(x-a), \varphi\right> = \left<\delta(x-a), \mathcal{F}\varphi \right> = \mathcal{F}\varphi(a) = \int\limits_{-\infty}^{\infty} e^{-2\pi i k a}\varphi(k) dk =\left<e^{-2\pi i x a}, \varphi \right>
.\]
\textbf{Obserwacja:}
\[
    \ddot{f} + \omega^2 f = \delta
.\]
\[
    \mathcal{F}\left( \ddot{f} + \omega^2 f = \delta \right)
.\]
\[
    \left( -2\pi i t \right)^2 \mathcal{F}f + \omega^2 \mathcal{F} f = \mathcal{F} \delta
.\]
\[
    -4\pi^2 t^2 \mathcal{F}f + \omega \mathcal{F}f = 1
.\]
\[
    \hat{f} = \frac{1}{\omega^2 - 4\pi^2t^2}
.\]
\begin{pytanie}
    A ile to $\mathcal{F}1$?
\end{pytanie}
\[
    \mathcal{F}1 = \int\limits_{-\infty}^{\infty} 1\cdot e^{-2\pi i k x}dk = -\frac{1}{2\pi i x}e^{-2\pi i k x}\Big|_{-\infty}^{+\infty} = ????
.\]
Tego napisu nie wolno traktować w sensie transformaty funkcji. A dystrybucji?
\textbf{Wniosek:} $\mathcal{F}1$ należy rozumieć w sensie dystrybucyjnym, czyli
\[
    \left<\mathcal{F}1, \varphi \right> = \left<1, \mathcal{F}\varphi \right>
.\]
Pamiętamy, że
\[
    \mathcal{F}\left( f^{(n)} \right) = \left( 2\pi i x \right)^n \mathcal{F}\left( f \right)
.\]
Czyli
\[
    \mathcal{F}(f') = 2\pi i x \mathcal{F}(f)
.\]
Jeżeli $f = 1$, to
\[
    0 = \mathcal{F}(0) = 2\pi i x \mathcal{F}(1)
.\]
Czyli $x \hat{1} = 0$. Wiemy, że jeżeli $xT = 0$, to
\[
    T = C_\alpha \delta
.\]
Czyli
\[
    \hat{1} = C_\alpha \delta
.\]
Pozostało policzyć ile to jest $C_\alpha$.
Wiemy, że
\[
    \left<\mathcal{F}1, \varphi \right> = \left<1, \mathcal{F}\varphi \right>
.\]
\[
    \left<C_\alpha \delta, \varphi \right> = \left<1, \mathcal{F}\varphi \right>
.\]
\[
    C_\alpha\left<\delta, \varphi \right> = \left<1, \mathcal{F}\varphi \right>
.\]
W szczególności dla
\[
    \varphi = e^{-ax^2}, \quad \mathcal{F}(\varphi) = e^{-\frac{\pi^2x^2}{a}}\sqrt{\frac{\pi}{a}}
.\]
\[
    \left( \int\limits_{-\infty}^{\infty} e^{-\alpha x^2}dx = \sqrt{\frac{\pi}{\alpha}}   \right)
.\]
Jeżeli $\varphi = e^{-x^2}$, $a = 1$,  $\mathcal{F}(\varphi) = e^{-(\pi^2x^2)}\sqrt{\pi} $, to
\[
    C_\alpha \left<\delta, e^{-x^2} \right> = \sqrt{\pi} \left<1, e^{-\pi^2x^2} \right>
.\]
\[
    C_\alpha e^0 = \sqrt{\pi} \int\limits_{-\infty}^{\infty} 1\cdot e^{-\pi^2 x^2}dx
.\]
\[
    C_\alpha = \sqrt{\pi} \sqrt{\frac{\pi}{\pi^2}}  = 1
.\]
\[
    \mathcal{F}1 = \delta \implies \mathcal{F}\delta = 1
.\]
\begin{definicja}
    \[
        \check{f}(x) \overset{\text{def}}{=} f(-x).
    \]
\end{definicja}
\begin{tw}
    \[
        \hat{\hat{f}}(x) = \check{f}(x)
    .\]
\end{tw}
\begin{proof}
\begin{align*}
    \hat{\hat{f}} &= \int\limits_{-\infty}^{\infty} dk e^{-2\pi i kx}\int\limits_{-\infty}^{\infty} ds e^{-2\pi i s k} f(s) = \int\limits_{-\infty}^{\infty} dk \int\limits_{-\infty}^{\infty} ds e^{-2\pi i k(x+s)}f(s)=\\
    &= \int\limits_{-\infty}^{\infty} ds f(s) \int\limits_{-\infty}^{\infty} dk e^{-2\pi i k (x+s)} = \int\limits_{-\infty}^{\infty} ds f(s) \left<e^{-2\pi i k (x+s)}, 1 \right> = \\
    &= \int\limits_{-\infty}^{\infty} ds f(s) \delta(x+s) = f(-x)
.\end{align*}
\end{proof}
\begin{pytanie}
    A ile to będzie
    \[
        \sum_{n=-\infty}^{\infty} e^{-2\pi i x n}
    ?\]
\end{pytanie}
No tyle
\[
    T(x) = \sum_{n=-\infty}^{\infty} \hat{\delta}(x-n)
.\]
\end{document}
