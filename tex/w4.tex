\documentclass[../main.tex]{subfiles}
\graphicspath{
    {"../img/"}
    {"img/"}
}

\begin{document}
\subsection{Końcówka dowodu (Stokesa na kostce)}
\begin{proof}

mamy definicję ścianki:
\[
    \partial I = \sum_{j = 1}^n\sum_{\alpha = 0,1}(-1)^{\alpha+j}I_{(j,\alpha)}
,\]
dla $I^n\subset\mathbb{R}^n$, $\omega\in \Lambda^{n-1}(M)$, $\omega = f(x^1, \ldots, x^n) = dx^1\land \ldots\land dx^{i-1}\land dx^{i+1}\land \ldots\land dx^n$. Wtedy dla $x = (x^1,\ldots,x^n)$ i $d\tilde x = dx^1\ldots dx^{i-1}dx^{i+1}\ldots dx^n$

\begin{align*}
    &\int\limits_{I(j,\alpha)}\left<f(x) d\tilde x, \frac{\partial }{\partial x^1}, \ldots, \frac{\partial }{\partial x^{j-1}}, \frac{\partial }{\partial x^{j+1}}, \ldots, \frac{\partial }{\partial x^n} \right> = \\
    &= \delta_{ij} \int\limits_{I(i,\alpha)}f(x^1,\ldots,x^{i-1},\alpha, x^{i+1}, \ldots x^n)d\tilde x =\\
    &= \int\limits_0^1 dx^1 \ldots \int\limits_0^1 dx^{i-1} \int\limits_0^1 dx^{i+1} \ldots \int\limits_0^1 dx^n f(x^1,\ldots,x^{i-1},\alpha, x^{i+1}, \ldots, x^n) \overset{(\star)}{=}\\
        &\overset{(\star)}{=} \int\limits_0^1dx^1\ldots\int\limits_0^1dx^n f(x^1,\ldots, x^{i-1},\alpha, x^{i+1}, \ldots, x^n) =\\
        &=\int\limits_{I^n}f(x^1,\ldots,x^{i-1},\alpha, x^{i+1}, \ldots, x^n)
.\end{align*}
Przechodzimy do sumy
\begin{align*}
    \int\limits_{\partial I}\omega &= \sum_{j=1}^{n} \sum_{\alpha = 0,1} (-1)^{\alpha+j}\int\limits_{I(j,\alpha)}\omega =\\
    &= \sum_{\alpha = 0,1}(-1)^{\alpha + i}\int\limits_{I^n}f(x^1,\ldots,x^{i-1},\alpha,x^{j+1},\ldots,x^n) =\\
    &= (-1)^{i+0}\int\limits_{I^n}f(x^1,\ldots,x^{i-1}, 0, x^{i+1},\ldots,x^n) + \\
    &+ (-1)^{i+1}\int\limits_{I^n}f(x^1,\ldots,x^{i-1},1,x^{i+1},\ldots,x^n)
.\end{align*}
\begin{align*}
    d\omega &= \frac{\partial f}{\partial x^i} dx^i\land dx^1\land \ldots\land dx^{i-1}\land dx^{i+1}\land \ldots\land dx^n = \\
    &= (-1)^{i+1} \frac{\partial f}{\partial x^i} dx^1\land \ldots\land dx^{i-1}\land dx^i \land dx^{i+1} \land \ldots \land dx^n
.\end{align*}
Stąd
\begin{align*}
    &(-1)^{i+1} \int\limits_{I^n}\left< \frac{\partial f}{\partial x^1} dx^1,\ldots,dx^n, \frac{\partial }{\partial x^1} , \ldots, \frac{\partial }{\partial x^n}  \right> = \\
    &= (-1)^{i+1} \int\limits_0^1dx^1\ldots\int\limits_0^1 dx^i\ldots\int\limits_0^1dx^n \frac{\partial f}{\partial x^i} (x) =\\
    &= (-1)^{i+1} \int\limits_0^1dx^1\ldots\int\limits_0^1dx^{i-1}\int\limits_0^1dx^{i+1}\ldots\int\limits_0^1dx^n\cdot  \\
    &\cdot \left[ f(x^1,\ldots,x^{i-1},1,x^{i+1},\ldots,x^n) - f(x^1,\ldots,x^{i-1}, 0, x^{i+1}, \ldots, x^n) \right]  \\
    &= (-1)^{i+1}\int\limits_0^1dx^1\ldots\int\limits_0^1dx^i\ldots\int\limits_0^1dx^n\cdot \\
    &\cdot \left[ f(x^1,\ldots,x^{i-1},1,x^{i+1},\ldots,x^n) - f(x^1,\ldots,x^{i-1}, 0, x^{i+1}, \ldots, x^n) \right] = \\
    &= (-1)^{i+1} \int\limits_{I^n}\left[ f(x^1,\ldots,x^{i-1},1,x^{i+1},\ldots,x^n) - f(x^1,\ldots,x^{i-1},0,x^{i+1},\ldots,x^n) \right]
.\end{align*}
\[
LHS = RHS
.\]
\end{proof}
\textbf{Uwaga:} Większą kostkę (w sensie długości krawędzi) możemy zawsze podzielić na sumę zorientowanych wspólnie kostek $I^n$. Całki na tych ścianach kostek, które się stykają dadzą w efekcie zero.
\begin{przyklad}
    Niech $[a,b]\in \mathbb{R}^1$ i $f\in \Lambda^0\left( [a,b] \right) $. Wtedy twierdzenie Stokesa wygląda tak (xD):
    \[
        \int\limits_{\partial[a,b]} f= \int\limits_{[a,b]} df = \int\limits_a^b \left<\frac{\partial f}{\partial x}dx , \frac{\partial }{\partial x}  \right>dx = \int\limits_a^b \frac{\partial f}{\partial x} dx = f(b) - f(a)
    .\]
\end{przyklad}
\begin{przyklad}
    Niech $\gamma$ - krzywa na $M$, $\dim M = 3$, $f \in \Lambda^0M$.
     \[
         \int\limits_\gamma df= \int\limits_{\partial \gamma} f = f(B) - f(A)
    .\]
\end{przyklad}
\begin{przyklad}
    $\dim M = 2$, niech $\alpha = xydx + x^2dy$. Policzmy $\int\limits_{\partial S}\alpha$.
    \[
    \int\limits_{\partial S}\alpha = \int\limits_{C_1}\alpha + \int\limits_{C_2}\alpha + \int\limits_{C_3}\alpha
    ,\]
ale
    \[
        \int\limits_{C_1}\left<\varphi^\star\alpha, \frac{\partial }{\partial x}  \right> = 0
    ,\]
$\varphi$ - parametryzacja $C_1$. Jeżeli weźmiemy sobie
\[
\int\limits_{C_3}\left<\varphi_3^\star\alpha, - \frac{\partial }{\partial y}  \right> = 0
,\]
$\varphi_3$ - parametryzacja $C_3$.
\[
    C_2 = \left\{ \begin{bmatrix} \cos\theta\\ \sin\theta \end{bmatrix}, 0\le \theta \le \frac{\pi}{2}  \right\}
,\]
zatem $\varphi_2^\star\alpha$ przy $x = \cos\theta \implies dx = -\sin\theta d\theta$, $y = \sin\theta \implies dy = \cos\theta d\theta$, mamy
\[
    \varphi_2^\star \alpha = \cos\theta \sin\theta(-\sin\theta d\theta) + (\cos^2\theta)\cos\theta d\theta = \cos\theta(\cos^2\theta - \sin^2\theta)d\theta
.\]
\[
    \int\limits_{\partial S}\alpha = \int\limits_0^{\frac{\pi}{2}}d\theta \left<\cos\theta(\cos^2\theta - \sin^2\theta)d\theta, \frac{\partial }{\partial \theta}  \right>
,\]
ale np. tw. Stokesa: $\int\limits_{\partial S}\alpha = \int\limits_S d\alpha$.
\[
d\alpha = xdy\land dx + 2xdx\land dy = xdx\land dy
.\]
\begin{align*}
    \int\limits_{\Box}\left<xdx\land dy, \frac{\partial }{\partial x} , \frac{\partial }{\partial y}  \right> &= \int\limits_0^1dx\int\limits_0^{\sqrt{1-x^2} } x = \int\limits_0^1dx \cdot x\sqrt{1-x^2} =\\
    &= \left.\frac{2}{3}(1-x^2)^{\frac{3}{2}}\frac{(-1)}{2}\right|_0^1 = \frac{1}{3}
.\end{align*}
\end{przyklad}
\begin{przyklad}
    Niech $\alpha = \frac{-y}{x^2+y^2}dx + \frac{x}{x^2+y^2}dy\in \Lambda^1(M)$, $\partial K = \left\{ \begin{bmatrix} \cos\theta\\ \sin\theta \end{bmatrix} , 0\le \theta, 2\pi \right\} $
    \[
        \int\limits_{\partial K}\alpha = \int\limits_0^{2\pi}\left<\varphi^\star\alpha, \frac{\partial }{\partial \theta} \right>d\theta
    .\]
\[
    \varphi^\star \alpha = -\sin\theta(-\sin\theta)d\theta + \cos\theta\cos\theta d\theta = d\theta
.\]
Czyli mamy
\[
\int\limits_{\partial K}\alpha = \int\limits_0^{2\pi}d\theta = 2\pi
.\]
Ale z drugiej strony dla
    \begin{align*}
        d\alpha &= \left[ \left( -\frac{1}{x^2+y^2} + \frac{2y\cdot y}{(x^2+y^2)^2} \right) dy\land dx + \left( \frac{1}{x^2+y^2} - \frac{2x^2}{(x^2+y^2)^2} \right) dx\land dy \right] = \\
        &= \left( \frac{2}{x^2 + y^2} - \frac{2}{x^2+y^2} \right) dx\land dy = 0
    .\end{align*}
     wyjdzie, że twierdzenie Stokesa się złamało.
\end{przyklad}
Wiemy, że
\[
    \int\limits_{\gamma}df = \int\limits_{\partial\gamma}f = f(B) - f(A)
.\]
Niech $\alpha = x^2dx + xydy + 2dz$. $\alpha$ jest potencjalna, jeżeli
\[
    \underset{\eta\in \Lambda^0M}{\exists} d\eta = \alpha \implies d(d\eta) = 0
,\]
(rotacja gradientu równa zero)
\[
    \int\limits_{\gamma}\alpha = \int\limits_{\gamma}d\eta = \eta(B) - \eta(A)
.\]

\pagebreak
\begin{definicja}
    Niech $M$ - rozmaitość, $\dim M = n$,\\
    \[
        i_v : T_pM \times \Lambda^kM\to \Lambda^{k-1}M
    \]
    zdefiniowana następująco:
    \begin{enumerate}
        \item $i_v f = 0$, jeżeli $f\in \Lambda^0M$
        \item  $i_v dx^i = v^i$, jeżeli $v = v^1 \frac{\partial }{\partial x^1} + \ldots + v^i \frac{\partial }{\partial x^i} + \ldots + v^n \frac{\partial }{\partial x^n} $
        \item $i_v(\omega\land \theta) = i_v(\omega)\land \theta + (-1)^{st \omega} \omega\land i_v(\theta)$.
    \end{enumerate}
    Operację $i_v$ nazywamy iloczynem zewnętrznym i oznaczamy poprzez
    \[
        i_v(\omega) \overset{\text{ozn}}{=} v \lrcorner \omega
    .\]
\end{definicja}
\textbf{Obserwacja:} $i_v(i_v\omega) = 0 $ (w domu)
\begin{przyklad}
    Niech $v = x \frac{\partial }{\partial x} + y \frac{\partial }{\partial y} + z \frac{\partial }{\partial z} $,
    \[
    \omega = dx\land dy + dz \land dx
    .\]
\[
    v \lrcorner \omega = \left<dx, v \right> \land dy + (-1)^1 dx\left<dy,v \right> + \left<dz, v \right>\land dx + (-1)^1 dz\land \left<dx, v \right>
.\]
\end{przyklad}
\begin{przyklad}
    \[
     F = E^xdx\land dt + E^ydy\land dt + E^z dz\land dt + B^x dy\land dz + B^ydz\land dx + B^zdx\land dy
    .\]
\[
j = e \frac{\partial }{\partial t} + ev^x \frac{\partial }{\partial x}  + ev^y \frac{\partial }{\partial y} + ev^z \frac{\partial }{\partial z}
.\]
\[
    j \lrcorner  F = ?
\]
\end{przyklad}
\end{document}
