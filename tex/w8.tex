\documentclass[../main.tex]{subfiles}
\graphicspath{
    {"../img/"}
    {"img/"}
}

\begin{document}
    \subsection{W ostatnim odcinku}
    \[
        \int_\gamma \alpha = \int_\gamma \vec{A}\cdot \underbrace{\vec{t}_{st}dL}_{\vec{dL}}
    .\]
\[
    dA^\sharp = \left( \overbrace{\left( . \right) , - \left( . \right)}^{D_1}  \right)dx^2\land dx^3 + \ldots
.\]
\[
    \int_S dA^\sharp = \int D^1\left<dx^2\land dx^3, \frac{\partial }{\partial x^2} , \frac{\partial }{\partial x^3}  \right>dx^2dx^3 + \int D^2 dx^{3} dx^{1} + \int D^3 dx^{1} dx^{2}
.\]

Przypomnijmy sobie czym jest rotacja wektora (takiego fizycznego)
\[
    rot(\vec{A}) = \left( \star\left( d\vec{A}^\sharp \right)  \right) ^\flat
,\]
ale
\begin{align*}
    \star(dx^{2} \land dx^{3} ) &= g^{22}g^{33}\sqrt{g}dx^{1},\\
    \star(dx^{3} \land dx^{1} ) &= g^{11}g^{33}\sqrt{g} dx^{2} ,\\
    \star(dx^{1} \land dx^{2} ) &= g^{11}g^{22}\sqrt{g} dx^{3}
.\end{align*}
Więc
\[
    \star dA^\sharp = D^1 g^{22}g^{33}\sqrt{g} dx^{1} + D^2 g^{33}g^{11}\sqrt{g} dx^{2} + D^3 g^{11}g^{22}\sqrt{g} dx^{3}
.\]
\begin{align*}
    \left( \star dA^\sharp \right) ^\flat &= D^1 g^{11}g^{22}g^{33}\sqrt{g} \frac{\partial }{\partial x^1} + D^2 g^{22}g^{33}g^{11}\sqrt{g} \frac{\partial }{\partial x^2} + D^3 g^{33}g^{11}g^{22}\sqrt{g} \frac{\partial }{\partial x^3}  =\\
    &= D^1 \sqrt{g^{22}g^{33}} \sqrt{g^{11}} \frac{\partial }{\partial x^1} + D^2 \sqrt{g^{11}g^{33}} \sqrt{g^{22}} \frac{\partial }{\partial x^2} + D^3 \sqrt{g^{11}g^{22}} \sqrt{g^{33}} \frac{\partial }{\partial x^3}
.\end{align*}
Czyli dla $\vec{A}$ - wektor w bazie ortonormalnej jest
\[
    rot \vec{A} = \begin{bmatrix} D^1 \frac{1}{\sqrt{g_{22}g_{33}} }\\ D^2 \frac{1}{\sqrt{g_{11}g_{33}}}\\ D^3 \frac{1}{g_{11}g_{22}} \end{bmatrix}
.\]
ale $rot(\vec{A})\cdot \vec{n} = D^1 \frac{1}{g_{22}g_{33}}$, ale
\[
    \left( rot \vec{A}\cdot \vec{n} \right) \cdot d\vec{s} = D^1 \frac{1}{g_{22}g_{33}}\sqrt{g_{22}g_{33}}dx^{2} dx^{3}
,\]
zatem
\[
    \int_S dA^\sharp = \int_S (rot \vec{A}) \cdot \vec{n}  ds
.\]
Czyli teraz mamy tak
\[
    \int_\gamma A^\sharp = \int_\gamma \vec{A}\cdot \vec{t}_{st}dL
.\]
\[
\int_S dA^\sharp = \int_{\partial S}A^\sharp
.\]
\[
    \int_S\left( rot \vec{A} \right) \cdot \vec{n} ds = \int_{\partial S}\vec{A} \cdot \vec{t}_{st}dL
.\]
\begin{przyklad}
    $\dim M = 3$, $V \subset M$, $\dim V = 3$
     \[
    \int_{\partial V}\star A^\sharp = \int_V d\star A^\sharp
    .\]
\end{przyklad}
\begin{pytanie}
czym jest $\int_{\partial V}\star A^\sharp$?
\end{pytanie}
\[
    \star(dx^{1})\sqrt{g}g^{11}  dx^{2}\land dx^{3}
,\]
\[
    \star(dx^{2})\sqrt{g}g^{22}  dx^{3}\land dx^{1}
,\]
\[
    \star(dx^{3})\sqrt{g}g^{33}  dx^{1}\land dx^{2}
,\]
\textbf{Odpowiedź:}
\begin{align*}
    \star A^\sharp &= A^1g_{11}\sqrt{g^{11}} \sqrt{g} g^{11}dx^{2} \land dx^{3} + A^2 g_{22}\sqrt{g^{22}} \sqrt{g} g^{22} dx^{3} \land dx^{1} +\\
    &+A^3 g_{33}\sqrt{g^{33}} \sqrt{g} g^{33}dx^{1} \land dx^{2}
,\end{align*}
następuje cudowne skrócenie i jest
\[
A^1 \sqrt{g_{22}g_{33}}\quad dx^{2} \land dx^{3}  + A^2 \sqrt{g_{11}g_{33}} \quad dx^{3} \land dx^{1} + A^3 \sqrt{g_{11}g_{22}} \quad dx^{1} \land dx^{2}
.\]
Całka z tego interesu:
\begin{align*}
    \int_{\partial V}\star A^\sharp &= \int A^1 \sqrt{g_{22}g_{33}}\quad dx^{2} dx^{3} + \int A^2 \sqrt{g_{11}g_{33}}\quad dx^{3} dx^{1} + \\
&+ \int A^3 \sqrt{g_{11}g_{22}} \quad dx^{1}  dx^{2}
,\end{align*}
ale
\[
    \vec{A}\cdot \vec{n}\cdot ds = A^1 \sqrt{g_{22}g_{33}}\quad dx^{2} dx^{3}
.\]
Czyli ostatecznie
\[
    \int_{\partial V}\star A^\sharp = \int_{\partial V}\vec{A}\cdot \vec{n}ds
.\]
\begin{pytanie}
Jak wygląda $\int_Vd\star A^\sharp$?
\end{pytanie}
\begin{align*}
    \int_{V}d\star A^\sharp &=\\
    &= \int_{V} \Bigg< \left( A^1\sqrt{g_{22}g_{33}} \right)_{,1} + \left( A^2\sqrt{g_{11}g_{33}}  \right) _{,2} +\\
    &+ \left( A^3\sqrt{g_{11}g_{22}}  \right) _{,3}, dx^{1} \land dx^{2} \land dx^{3} , \frac{\partial }{\partial x^1} , \frac{\partial }{\partial x^2} , \frac{\partial }{\partial x^3} \Bigg> dx^{1} dx^{2} dx^{3}
.\end{align*}
Dywergencja to było coś takiego:
\[
    div \vec{A} = \star d\left( \star A^\sharp \right)
,\]
wiemy, że
\[
    \star\left( dx^{1} \land dx^{2} \land dx^{3}  \right) = \sqrt{g} g^{11}g^{22}g^{33} = \sqrt{g^{11}g^{22}g^{33}}
,\]
więc
\[
    div \vec{A} \sqrt{g_{11}g_{22}g_{33}}\quad dx^{1} dx^{2} dx^{3}= div\vec{A}\quad dV
.\]
Zatem ze zdania
\[
\int_{\partial V}\star A^\sharp = \int_V d\star A^\sharp
\]
wiemy, że
\[
    \int_{\partial V}\vec{A}\cdot \vec{n} ds = \int_V div \vec{A} \quad dV
.\]

\subsection{Analiza Zespolona}
(podobno bardzo przyjemny dział analizy)\\
(rys 8-2)

Można się zastanowić nad taką funkcją:
\[
f : \mathbb{R}\to \mathbb{C}
,\]
\[
    f(t) = e^{iat};\quad a > 0
,\]
(kółko)
\[
    f(t) = e^{bt}e^{iat};\quad a, b > 0
.\]
(spiralka)
\begin{definicja}
    Niech $\mathcal{O}\subset\mathbb{C}$, $\mathcal{O}$ - otwarty.
    $f : \mathcal{O}\to \mathbb{C}$. \\
    Mówimy, że $f$ jest holomorficzna na $\mathcal{O}$ jeżeli $\underset{z\in \mathcal{O}}{\forall}$ istnieje granica
    \[
        \lim_{h \to 0}\frac{f(z+h) - f(z)}{h} \overset{\text{def}}{=} f'(z)
    ,\]
gdzie $f'(z)$ jest funkcją ciągłą.
\end{definicja}
\textbf{Uwaga:} jeżeli nie zostanie to podkreślone, to wszystkie niezbędne struktury przenosimy z $\mathbb{R}^2$.\\
\textbf{Uwaga:} dowolną funkcję z $\mathbb{C}$ możemy zapisać jako $f(z) = P(x,y) + Q(x,y)\cdot i$, gdzie $z = x + iy$ a $P(x,y): \mathbb{R}^2\to \mathbb{R}^1$, $Q(x,y) : \mathbb{R}^2 \to \mathbb{R}^1$
\begin{przyklad}
    $f(z) = \cos x + i \sin(xy)$, $z = x + iy$
\end{przyklad}

\begin{pytanie}
    Co to znaczy różniczkowalność?
\end{pytanie}
ma istnieć granica (dla $h\in \mathbb{R}$):
\begin{align*}
    \lim_{h \to 0} \frac{f(z+h) - f(z)}{h} &= \lim_{h \to 0} \frac{P(x+h, y) + iQ(x + h, y) - P(x,y) - iQ(x,y)}{h} =\\
    &= \frac{\partial P}{\partial x} + i \frac{\partial Q}{\partial x}
.\end{align*}
Ale jeżeli np. $h = it$, to wtedy
\begin{align*}
    \lim_{h \to 0}\frac{f(z+h) - f(z)}{h} &= \lim_{t \to 0}\frac{P(x, y+t) - P(x,y)}{it} + i \frac{Q(x, y+t) - Q(x,y)}{it} =\\
    &= \frac{1}{i} \frac{\partial P}{\partial y} + \frac{\partial Q}{\partial y} = \frac{\partial Q}{\partial y} - i \frac{\partial P}{\partial y}
.\end{align*}
Czyli jeżeli $f$ - holomorficzna, to znaczy, że (wzory Cauchy-Riemanna)
\begin{align*}
    \frac{\partial P}{\partial x} &= \frac{\partial Q}{\partial y}\\
    \frac{\partial Q}{\partial y} &= - \frac{\partial P}{\partial x}
.\end{align*}
\begin{przyklad}
    (jak mogła by wyglądać funkcja różniczkowalna?)
    \[
        f(z) = \underbrace{x}_{P(x,y)} - i\underbrace{y}_{Q(x,y)}
    .\]
Czy $f$ jest różniczkowalna?
\[
\frac{\partial P}{\partial x} = 1,\quad \frac{\partial P}{\partial y} = 0,\quad \frac{\partial Q}{\partial x}  = 0,\quad \frac{\partial Q}{\partial y} = -1
,\]
czyli coś nie gra, bo jak to ma nie być różniczkowalne
\end{przyklad}
\begin{przyklad}
    \[
        \alpha = Q(x,y)dx + P(x,y)dy
    ,\]
gdzie $P$, $Q$ są takie, że $f(z) = P(x,y) + iQ(x,y)$ jest holomorficzna.
\[
    d\alpha = \frac{\partial Q}{\partial y} dy\land dx + \frac{\partial P}{\partial x} dx\land dy = \left( \frac{\partial P}{\partial x} - \frac{\partial Q}{\partial y}  \right) dx\land dy = 0
.\]
\end{przyklad}
\begin{pytanie}
    Niech $f(z) = P(x,y) + iQ(x,y)$, f - holomorficzna. Co ciekawego można powiedzieć o zbiorach
    \[
        P_c = \left\{ (x,y)\in \mathbb{R}^2,\quad P(x,y) = c\in\mathbb{R} \right\}
    .\]
    \[
        Q_d = \left\{ (x,y)\in \mathbb{R}^2,\quad Q(x,y) = d\in \mathbb{R} \right\}
    .\]
\end{pytanie}


\end{document}
