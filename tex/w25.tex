\documentclass[../main.tex]{subfiles}
\graphicspath{
    {"../img/"}
    {"img/"}
}

\begin{document}
\begin{przyklad}
    \[
        \ddot{x} + \omega^2 x = f(t)\quad \partial_{t^2}u + \omega^2 u = \delta(t-a)
    .\]
\[
    \left<\partial_{t^2}u + \omega^2 u , \varphi \right> = \left<\delta(t-a), \varphi \right>
.\]
Ale to jest całka po $a$, czy po $t$?
Dla
\begin{align*}
    x(t) &= \int\limits_{-\infty}^{\infty} u(t,a)f(a)da\\
    \dot{x}(t) &= \int\limits_{-\infty}^{\infty} \partial_t u(t,a)f(a)da\\
    \ddot{x}(t) &= \int\limits_{-\infty}^{\infty} \partial_{t^2}u(t,a)f(a)da
.\end{align*}
Czyli
\[
    \ddot{x} + \omega^2 x = \int\limits_{-\infty}^{\infty} \left( \partial_{t^2}u(t,a) + \omega^2 u(t,a) \right) f(a)da = \int\limits_{-\infty}^{\infty} \delta(t-a)f(a)da = \underset{(\star)}{-f(t)}
.\]
\textbf{Uwaga:} Jak pozbyć się minusa w $(\star)$?\\
    Trzeba rozstrzygnąć problem
        \begin{enumerate}[a)]
            \item $\partial_{t^2}u + \omega^2 u = -\delta(t-a)$
            \item $\partial_{t^2}u + \omega^2 u = -\delta(a-t)$
            \item $\partial_{t^2}u + \omega^2 u = -\delta(t)$, $x(t) = (u\star f)$
        \end{enumerate}
    Funkcja $u$ nazywa się czasem {\color{green}funkcją Greena}.
\end{przyklad}
\begin{przyklad}
    Czasem problem
    \[
        L \varphi = \rho(x)
    \]
    możemy rozwiązać problemem
    \[
        L u = \delta
    .\]
\end{przyklad}
\begin{przyklad}
    Wiemy, że $div(E) = \rho(x)$. Mamy też napis $E = - \nabla \varphi$.\\
    Czyli
    \[
        -div(grad(\varphi)) = \rho(x)
    .\]
\[
    \Delta \varphi = -\rho(x)
.\]
Spróbujemy poradzić sobie z minusem. Traktujemy to równanie jako dystrybucyjne.
\[
    \Delta u = \delta \longrightarrow \left<\Delta u, \varphi \right> = \left<\delta, \varphi \right>
.\]
Wtedy $\varphi = (u\star \rho)$
\[
    \varphi(x_0) = \left( \frac{1}{\left\Vert x \right\Vert }\star \rho \right) = \int\limits_{V}\frac{\rho(x')d^3x'}{\left\Vert x_0 - x' \right\Vert }
.\]
\end{przyklad}
\subsection{\color{green}Wzór Greena}
\begin{tw}
Niech $U, V: \mathbb{R}^2 \to \mathbb{R}$. $U, V \subset \mathcal{C}^2(\mathbb{R}^3)$. Niech $M$ - rozmaitość, $M\subset\mathbb{R}^3$. Wówczas
\[
    \int\limits_M \left(u \Delta v - v \Delta u\right)dV = \int\limits_{\partial M}\left( u \frac{\partial }{\partial \eta} v - v \frac{\partial }{\partial \eta} u\right)dS
.\]
Gdzie $\frac{\partial }{\partial \eta} v$ - pochodna wzdłuż wektora normalnego do powierzchni $\partial M$. Czyli $\frac{\partial }{\partial \eta} v = \left( \nabla u \right)\cdot \eta$
\end{tw}
\begin{proof}
    Wiemy, że jeżeli $\omega\in\Lambda^2(\mathbb{R}^3)$, to
    \[
        \int\limits_M d\omega = \int\limits_{\partial M}\omega
    .\]
Zatem, jeżeli $\omega = \star du$, to znaczy, że
\begin{equation}
    \label{eqn:w25-1}
    \int\limits_M d(v\star du) = \int\limits_{\partial M}v\star du\tag{A}
\end{equation}
A jeżeli $\omega = u\star dv$
\begin{equation}
    \label{eqn:w25-2}
     \int\limits_M d(u\star dv) = \int\limits_{\partial M}u\star dv\tag{B}
\end{equation}
    Odejmując (\ref{eqn:w25-2}) od (\ref{eqn:w25-1}) otrzymamy
    \[
        \int\limits_M du\land \star dv + u d\star dv - dv\land \star du - v d\star du = \int\limits_{\partial M}u\star dv - v\star du
    .\]
Zauważmy, że jeżeli $A = A_xdx + A_ydy + A_zdz$ i  $B = B_xdx + B_ydy + B_zdz$, to
 \[
    \star A = A_xdy\land dz + A_y dz\land dx + A_z dx\land dy
.\]
\[
    \star B = B_xdy\land dz + B_y dz\land dx + B_z dx\land dy
.\]
Zatem
\[
    A\land \star B = \left( A_xB_x + A_yB_y + A_zB_z \right) dx\land dy\land dz
.\]
\[
    B\land \star A = \left( B_xA_x + B_yA_y + B_zA_z \right) dx\land dy\land dz
.\]
Oznacza to, że
\[
    du\land \star dv - dv\land\star du = 0
.\]
Zatem
\[
    \int\limits_M ud\star dv - vd\star du = \int\limits_M u\Delta v - v \Delta u = \int\limits_{\partial M} u\star dv - v\star du
.\]
Zauważmy, że jeżeli $v(x,y,z): \mathbb{R}^3\to \mathbb{R}^1$,
\begin{align*}
    dv &= v_xdx + v_ydy + v_zdz\\
    \star dv &= v_xdy\land dz + v_ydz\land dx + v_zdy\land dx
.\end{align*}
Weźmy sobie kostkę z $\mathbb{R}^3$. Wtedy
\[
    \int\limits_{\partial M}\star dv = \sum_{i=1}^{6} \int \left<\star dv, \frac{\partial }{\partial x^k} , \frac{\partial }{\partial x^l}  \right> = \int\limits_{\partial M}\left( \nabla v \right) n dS
.\]
Zatem, przechodząc od form do całek po funkcjach, otrzymujemy
\[
    \int\limits_M\left( u\Delta v - v\Delta u \right) dV = \int\limits_{\partial M}\left( u \frac{\partial v}{\partial \eta} - v \frac{\partial u}{\partial \eta}  \right) dS
.\]
\end{proof}
\begin{stw}
    Jeżeli $r = \sqrt{x^2 + y^2 + z^2}$, to w sensie dystrybucyjnym
    \[
        \Delta \frac{1}{r} = \delta \longleftarrow \left<\Delta \frac{1}{r}, \varphi \right> = \left<\delta, \varphi \right>, \left<\delta, \varphi \right> = \varphi(0)
    .\]
\end{stw}
\begin{proof}
    Zauważmy, że
    \[
        \left<\Delta\left( \frac{1}{r} \right) , \varphi \right> = \left<\nabla\cdot \nabla\left( \frac{1}{r} \right) , \varphi \right> = -\left<\nabla\left( \frac{1}{r} \right) , \nabla \varphi \right> = \left<\frac{1}{r}, \Delta \varphi \right>
    .\]
Chcemy pokazać, że
\[
    \underset{\varphi\in D}{\forall} \left<\frac{1}{r}, \Delta \varphi \right> = \left<\delta, \varphi \right>
.\]
Od lewej:
\[
    \left<\frac{1}{r}, \Delta \varphi \right> = \int\limits_{\mathbb{R}^3}\left( \frac{1}{r}\Delta \varphi \right) dV
.\]
Wiemy, że $\varphi$ ma nośnik zwarty, więc zamiast po $\mathbb{R}^3$, możemy całkować po objętości $V$ (Jak $V$ ma się do nośnika $\varphi$, to zobaczymy).
\[
    \int\limits_{V}\left( \frac{1}{r}\Delta \varphi \right) dV = \lim_{\varepsilon \to 0}\int\limits_{V\setminus K(0,\varepsilon)}\left( \frac{1}{r}\Delta \varphi \right) dV
.\]
Odpalamy {\color{green} wzór Greena}
    Niech $u = \frac{1}{r}$, $v = \varphi$, $M = V\setminus K(0,\varepsilon)$. Wtedy
    \begin{equation}
        \label{eqn:w25-3}
        \int\limits_{V\setminus K(0,\varepsilon)}\left( \frac{1}{r}\Delta \varphi - \varphi\Delta \frac{1}{r} \right)dV = \int\limits_{\partial\left( V\setminus K(0,\varepsilon) \right) }\left( \frac{1}{r}\frac{\partial \varphi}{\partial \eta} - \varphi \frac{\partial }{\partial \eta} \left( \frac{1}{r} \right)  \right) dS\tag{$\clubsuit$}
    \end{equation}
    Zauważmy, że $\Delta \frac{1}{r}$, gdy $(x,y,z)\in V\setminus K(0,\varepsilon)$ wynosi
    \begin{align*}
        &\frac{\partial }{\partial x} \left( \frac{\partial }{\partial x} \frac{1}{r} \right) + \frac{\partial }{\partial y} \left( \frac{\partial }{\partial y} \frac{1}{r} \right) + \frac{\partial }{\partial z} \left( \frac{\partial }{\partial z} \frac{1}{r} \right) = \\
        &= -\frac{1}{r^3} - \frac{1}{r^3} - \frac{1}{r^3} + \frac{3x^2}{r^5} + \frac{3 y^2}{r^5} + \frac{3z^2}{r^5}\\
        &=  0 \\
    .\end{align*}
    Zatem
    \[
        \int\limits_{V\setminus K(0,\varepsilon)} \frac{1}{r}\Delta \varphi = \int\limits_{\partial\left( V\setminus K(0,\varepsilon) \right) }\left( \frac{1}{r}\frac{\partial \varphi}{\partial \eta} - \varphi \frac{\partial }{\partial \eta} \left( \frac{1}{r} \right)  \right)dS
    .\]
Ale
\[
    \int\limits_{\partial\left( V\setminus K(0,\varepsilon) \right) }\left(  \right) = \int\limits_{\partial V}\left( \right) + \int\limits_{\partial K(0,\varepsilon)}\left(  \right)
.\]
(\textbf{uważać na orientację})\
    Wybierzemy $V$ na tyle duże, żeby nośnik $\varphi\subset V$. Oznacza to, że $\varphi(x)\Big|_{x=\partial V}= 0$ i $\frac{\partial \varphi}{\partial x} \Big|_{x=\partial V} = 0$.
    Zatem
    \[
        \int\limits_{V\setminus K(0,\varepsilon)}\frac{1}{r}\Delta \varphi = - \int\limits_{\partial K(0,\varepsilon)}\left( \frac{1}{r}\frac{\partial \varphi}{\partial \eta} - \varphi \frac{\partial }{\partial \eta} \left( \frac{1}{r} \right)  \right) dS
    .\]
Ale znamy twierdzenie o wartości średniej
\[
    \int\limits_{\partial K(0,\varepsilon)}\frac{1}{r} \frac{\partial \varphi}{\partial \eta} dS = \frac{\partial \varphi}{\partial \eta} \Big|_{c} \cdot \int\limits_{\partial K(0,\varepsilon)}\frac{1}{r}dS = \frac{\partial \varphi}{\partial \eta} \Big|_c \cdot 4\pi \cdot \frac{1}{\varepsilon} \cdot \varepsilon^2 \underset{\varepsilon\to 0}{\longrightarrow} 0
.\]
Teraz mamy
\begin{align*}
    \int\limits_{\partial K(0,\varepsilon)}\varphi \frac{\partial }{\partial \eta} \left( \frac{1}{r} \right) dS &= \varphi_{(c)}\int\limits_{\partial K(0,\varepsilon)}\frac{\partial }{\partial \eta} \left( \frac{1}{r} \right) = \varphi_{(c)}\int\limits_{\partial K(0,\varepsilon)} \frac{\partial }{\partial r} \left( \frac{1}{r} \right) =\\
    &= \varphi_{(c)}\int\limits_{\partial K(0,\varepsilon)} -\frac{1}{r^2} = -\varphi_{(c)}\cdot 4\pi \cdot \frac{1}{\varepsilon^2}\varepsilon^2 = -4\pi \varphi_{(c)} \underset{\varepsilon\to 0}{\longrightarrow} -4\pi \varphi(0)
.\end{align*}
\[
    \left<\Delta \frac{1}{r}, \varphi \right> = \lim_{\varepsilon \to 0}\int\limits_{V\setminus K(0,\varepsilon)}\frac{1}{r}\Delta \varphi= -4\pi \varphi(0) = -4\pi \left<\delta, \varphi \right>
.\]
\[
    \Delta\left( \frac{1}{r} \right) = -4\pi \delta
.\]
\end{proof}
\end{document}
