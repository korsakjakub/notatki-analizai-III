\documentclass[../main.tex]{subfiles}
\graphicspath{
    {"../img/"}
    {"img/"}
}

\begin{document}
\begin{definicja}
    Niech $\mathcal{O}\subset\mathbb{R}^n$. Zbiór $\mathcal{O}$ nazywamy ściągalnym (gwiaździstym), jeżeli istnieje $p\in \mathcal{O}$ i odwzorowanie $h(p,x,t)$ takie, że
    \[
        \underset{x\in\mathcal{O}}{\forall}\quad \begin{matrix}h(p,x,0) = p\\ h(p,x,1) = x\end{matrix},\quad \underset{t\in[0,1]}{\forall}  h(p,x,t)\in \mathcal{O}, \quad h(p,x,t) \text{ - ciągła}
    .\]
\end{definicja}
\begin{tw}
    (Lemat Poincare)\\
    Niech
    \[
        \begin{pmatrix}\mathcal{O} \text{ - zbiór ściągalny}\\ \dim \mathcal{O} = n\\ \omega\in\Lambda^{p-1}(\mathcal{O})\\ d\omega = 0\end{pmatrix} \implies \begin{pmatrix} \underset{\eta}{\exists}, d\eta = \omega \\ \eta\in\Lambda^{p-1}(\mathcal{O}) \end{pmatrix}
    .\]
\end{tw}
\begin{proof}
    Załóżmy, że zbiór $\mathcal{O}$ jest zbiorem gwiaździstym, czyli
    \[
        \underset{p\in\mathcal{O}}{\exists}\quad \underset{x\in \mathcal{O}}{\forall}\quad \begin{pmatrix}\text{zbiór punktów postaci}\\ pq_1 + xq_2: q_1+q_2 = 1, q_1,q_2 > 0\end{pmatrix}\begin{pmatrix} \text{jest zawarty w }\mathcal{O} \end{pmatrix}
    .\]
\textbf{Obserwacja:} gdyby istniał operator
    \[
        T: \Lambda^p(\mathcal{O}) \to \Lambda^{p-1}(\mathcal{O}),\quad p = 1,2,\ldots,n-1,
    \]
    taki, że
\[
Td + dT = id
,\]
to twierdzenie byloby prawdziwe. (bo dla $\omega\in \Lambda^p(\mathcal{O})$ mielibyśmy $Td(\omega) + d(T\omega) = \omega$).

    Więc, gdy
    \[
        d\omega = 0
    ,\]
to
    \[
        d(T\omega) = \omega
    ,\]
    czyli przyjmując
 \[
\eta = T\omega
,\]
otrzymujemy
\[
    d(\eta_i) = \omega
.\]
Łatwo sprawdzić, że operator
\[
    T_1(\omega) = \int_0^1\left(t^{p-1}x \lrcorner \omega(tx)\right)
,\]
$x = x^1 \frac{\partial }{\partial x^1} + x^2 \frac{\partial }{\partial x^2} + \ldots + x^n \frac{\partial }{\partial x^n}$ spełnia warunek $Td + dT = id$.
 \begin{przyklad}
     $\omega \in \Lambda^1(M)$, $\dim M = 3$, $\omega = xdx + ydy + zdz$. Wówczas, gdy ($\hat{x} = x \frac{\partial }{\partial x} + y \frac{\partial }{\partial y} + z \frac{\partial }{\partial z}$ ) jest
     \begin{align*}
         T(\omega) &= \int_0^1 t^{1-1} \left<\underbrace{(xt)dx + (yt)dy + (zt)dz}_{\omega(tx)},\quad \hat{x} \right>dt = \\
         &= \int_0^1t^0\left( tx^2 + ty^2 + tz^2 \right) dt = \frac{1}{2}\left( x^2 + y^2 + z^2 \right) = \eta.
     \end{align*}
     Zauważamy, że $d\eta = \omega$ i działa (dla takiego radialnego pola wektorowego znaleźliśmy potencjał).
\end{przyklad}
\begin{przyklad}
    $\omega = xdx\land dy + ydy\land dz + zdx\land dz$,  $\omega\in \Lambda^2(M)$, $\dim M = 3$.
    Co to jest $T\omega$?
    \begin{align*}
        T\omega &= \int_0^1t^{2-1} x \lrcorner \left( xt dx\land dy + yt dy\land dz + zt dx\land dz \right) dt = \\
        &= \int_0^1 t^1\left( xtx dy - xt ydx + yt ydz - ytzdy + ztxdz - ztzdx \right)dt = \\
        &= \frac{1}{3}\left( x^2dy - xydx + y^2dz - yzdy + zxdz - z^2dx \right) = \eta
    .\end{align*}
\end{przyklad}
Niech
\[
    T\omega = \int_0^1 t^{p-1} x \lrcorner \omega(tx)dx
,\]
gdzie $x = x^1 \frac{\partial }{\partial x^1} + \ldots + x^n \frac{\partial }{\partial x^n} $.\\
Chcemy pokazać, że
\[
dT\omega + Td\omega = \omega
,\]
gdzie
\[
    \omega(x) = \sum_{i_1, \ldots, i_p}\omega_{i_1, \ldots, i_p}(x^1, \ldots, x^n)dx^{i_1}\land \ldots\land dx^{i_p}
.\]
\[
    \omega = \overset{\omega_{12}}{x} d\overset{i_1 = 1}{x}\land d\overset{i_2 = 2}{y} + \overset{\omega_{23}}{y}d\overset{i_1 = 2}{y}\land d\overset{i_2 = 3}{z} + \overset{\omega_{13}}{z}d\overset{i_1 = 1}{x}\land d\overset{i_2 = 3}{z}
.\]
\[
    d\omega = \sum_{i_1,\ldots,i_p}\sum_{j=1}^n \frac{\partial \omega(x^1,\ldots,x^n)}{\partial x^j} dx^j\land dx^{i_1}\land \ldots \land dx^{i_p}
.\]
Liczymy
\begin{align*}
    Td\underset{p+1\text{ forma}}{\omega} &= \int_0^1t^{p+1-1}\left( x^1 \frac{\partial }{\partial x^1} + \ldots + x^n \frac{\partial }{\partial x^n}  \right)\lrcorner \frac{\partial \omega(tx^1,\ldots,tx^n)}{\partial x^j} dx^j\land dx^{i_1}\land \ldots\land dx^{i_p} = \\
    &= \sum_{j=1}^n \int_0^1 t^p dt\frac{\partial \omega(tx^1,\ldots,tx^n)}{\partial x^j}x^jdx^{i_1}\land \ldots\land dx^{i_p} +\\
    &+\sum_{j=1}^n\sum_{\alpha = 1}^p \int_0^1 t^p dt \frac{\partial \omega(tx^1,\ldots,tx^n)}{\partial x^j}x^{i_\alpha}dx^{i_1}\land\underset{\text{brak }dx^{i_\alpha}}{\ldots} \land dx^{i_p}(-1)^{\alpha}
.\end{align*}
\begin{align*}
    T\omega &= \int_0^1 t^{p-1} \left(x^1 \frac{\partial }{\partial x^1} + \ldots + x^n \frac{\partial }{\partial x^n} \right)\lrcorner \omega_{i_1,\ldots,i_p} (tx^1, \ldots, tx^n) dx^{i_1}\land \ldots\land dx^{i_p} = \\
    &= \sum_{k = 1}^{}\int_0^1dt\quad t^{p-1} \omega_{i_1,\ldots,i_p}(tx^1,\ldots,tx^n)x^kdx^{i_1}\land \underset{\text{bez }dx^{i_k}}{\ldots}\land dx^{i_p}(-1)^{k+1}.\\
    dT\omega &= \sum_{k=1}^p \int_0^1 dt t^{p-1} \omega_{i_1,\ldots,i_p}(tx^1,\ldots,tx^n)dx^{i_1}\land \ldots\land dx^{i_p} +\\
    &+\sum_{k=1}^p \int_0^1 dt t^{p-1} \sum_{\alpha = 1}^n \frac{\partial \omega_{i_1,\ldots,i_p}(tx^1,\ldots,tx^n)}{\partial x^\alpha}\cdot t\cdot  x^{i_k}dx^\alpha\land dx^{i_1}\land \ldots\land dx^{i_p}
.\end{align*}
Zatem dodajemy do siebie $Td\omega + dT\omega$ i wychodzi
\begin{align*}
    Td\omega + dT\omega &= \sum_{j = 1}^n\int_0^1dt\cdot t^p \frac{\partial \omega_{i_1,\ldots,i_p}(tx^1,\ldots,tx^n)}{\partial x^j} x^j dx^{i_1}\land \ldots\land dx^{i_p} +\\
    &+ \int_0^1dt p\cdot t^{p-1}\omega_{i_1,\ldots,i_p}(tx^1,\ldots,tx^n)dx^{i_1}\land \ldots\land dx^{i_p} + \underset{\text{równa się zero}}{\left( . \right) + \left( . \right)} = \\
    &= \int_0^1dt\left(\frac{d}{dt}\left( t^p \omega(tx^1,\ldots,tx^n)dx^{i_1}\land \ldots\land dx^{i_p}\right)\right) =\\
    &= t^p\left.\left(\omega(tx^1,\ldots,tx^n)dx^1\land\ldots\land dx^p\right)\right|_{t = 0}^{t = 1} = \omega
.\end{align*}

\end{proof}
\end{document}
