\documentclass[b5paper]{memoir}
\usepackage[utf8]{inputenc}
%\usepackage[margin=1in]{geometry}
\usepackage{mdframed}
\usepackage{xcolor}
\usepackage{subfiles}
\usepackage{subcaption}
\usepackage{enumerate}
\usepackage{graphicx}
\usepackage{wrapfig}
\usepackage{amsmath}
\usepackage{amsfonts}
\usepackage{amssymb}
\usepackage{amsthm}
\usepackage{polski}
\usepackage{import}
\usepackage{xifthen}
\usepackage{pdfpages}
\usepackage{transparent}
\usepackage{fancyhdr}
\usepackage{lastpage}
\usepackage{titlesec}

\pagestyle{fancy}
\fancyhead{}
\fancyfoot{}
\fancyhead[RO, LE]{\thepage}
\fancyhead[C]{Analiza III}
\let\LaTeXStandardTableOfContents\tableofcontents

\renewcommand{\tableofcontents}{%
\begingroup%
\renewcommand{\bfseries}{\relax}%
\LaTeXStandardTableOfContents%
\endgroup%
}%

\chapterstyle{dash}


\titleformat{\chapter}[block]
  {\normalfont\LARGE}{Wykład \thechapter.}{1em}{\Large}
\titlespacing*{\chapter}{0pt}{-19pt}{32pt}

\newcommand{\incfig}[1]{%
    \def\svgwidth{0.8\columnwidth}
    \IfFileExists{img/#1.pdf_tex}{
        \import{"img/"}{#1.pdf_tex}
    }
    {
        \import{"../img/"}{#1.pdf_tex}
    }
}

\newmdtheoremenv{tw}{Twierdzenie}
\newmdtheoremenv{stw}{Stwierdzenie}
\newmdtheoremenv{definicja}{Definicja}
\newtheorem{pytanie}{Pytanie}
\newtheorem{przyklad}{Przykład}
\newtheorem{uwaga}{Uwaga}

\DeclareMathOperator{\ilwewn}{\lrcorner\,}

\DeclareMathOperator{\Res}{Res}

\title{\Huge \textbf{Wykłady z Analizy III}}
\author{Jakub Korsak}
\date{X 2019 - II 2020}
\begin{document}

\frontmatter
\maketitle
\pagebreak
\tableofcontents

\chapter{Wstęp}
Niniejszy dokument zawiera moje notatki z wykładu Analiza III wygłoszonego przez dr Marcina Kościeleckiego na Wydziale Fizyki UW w semestrze zimowym roku akademickiego 2019/2020.
\mainmatter

\chapter{04.10.2019, \textit{przypomnienie i całka z jednoformy}}
\subfile{tex/w1.tex}
\chapter{07.10.2019, \textit{całka po kostce, rozmaitości zorientowane i prawie twierdzenie Stokesa}}
\subfile{tex/w2.tex}
\chapter{11.10.2019, \textit{wstęga Moebiusa i dowód twierdzenia Stokesa (1/2)}}
\subfile{tex/w3.tex}
\chapter{14.10.2019, \textit{dowód twierdzenia Stokesa (2/2), agitacja na temat lematu Poincare i iloczyn wewnętrzny}}
\subfile{tex/w4.tex}
\chapter{18.10.2019, \textit{brzeg rozmaitości i dalsza agitacja lematu Poincare}}
\subfile{tex/w5.tex}
\chapter{21.10.2019, \textit{dowód lematu Poincare, przykłady}}
\subfile{tex/w6.tex}
\chapter{25.10.2019, \textit{domkniętość i zupełność formy, długość krzywej i zastosowania twierdzenia Stokesa}}
\subfile{tex/w7.tex}
\chapter{28.10.2019, \textit{zastosowania twierdzenia Stokesa, holomorficzność funkcji i wzory Cauchy-Riemanna}}
\subfile{tex/w8.tex}
\chapter{04.11.2019, \textit{warunek Cauchy-Riemanna, wzór Cauchy i twierdzenie Liouville (1/2)}}
\subfile{tex/w9.tex}
\chapter{08.11.2019, \textit{twierdzenie Liouville (2/2), Zasadnicze Twierdzenie Algebry i początek Szeregów Laurenta}}
\subfile{tex/w10.tex}
\chapter{15.11.2019, \textit{zabawa z Szeregiem Laurenta, związki z szeregiem Taylora}}
\subfile{tex/w11.tex}
\chapter{22.11.2019, \textit{przedłużenie analityczne funkcji punkty osobliwe i bieguny}}
\subfile{tex/w12.tex}
\chapter{18.11.2019, \textit{punkt izolowany, osobliwość istotna, twierdzenie o residuach}}
\subfile{tex/w13.tex}
\chapter{25.11.2019, \textit{fajność residuów i Transformata Legendre geometrycznie}}
\subfile{tex/w14.tex}
\chapter{29.11.2019, \textit{Lemat Jordana, funkcja wokół punktu istotnie osobliwego i twierdzenie Weierstrass}}
\subfile{tex/w15.tex}
\chapter{02.12.2019, \textit{sumowanie szeregów}}
\subfile{tex/w16.tex}
\chapter{06.12.2019, \textit{twierdzenie Rouche, Zasadnicze Twierdzenie Algebry v2.0, sumowanie szeregów v2.0, residuum w $+\infty$ (1/3)}}
\subfile{tex/w17.tex}
\chapter{09.12.2019, \textit{przygotowanie do twierdzenia Kasner-Arnold, krzywizna, odwzorowania konforemne, residuum w $+\infty$ (2/3)}}
\subfile{tex/w18.tex}
\chapter{13.12.2019, \textit{twierdzenie Kasner-Arnold}}
\subfile{tex/w19.tex}
\chapter{16.12.2019, \textit{residuum w $+\infty$ (3/3) + super twierdzenie, transformata Fouriera}}
\subfile{tex/w20.tex}
\chapter{20.12.2019, \textit{własności transformaty Fouriera i transformata odwrotna}}
\subfile{tex/w21.tex}
\chapter{09.01.2020, \textit{Splot, wchodzenie z granicą pod całkę i równanie przewodnictwa}}
\subfile{tex/w22.tex}
\chapter{10.01.2020, \textit{Iloczyn skalarny, unitarność transformaty Fouriera, nierówność Heisenberga}}
\subfile{tex/w23.tex}
\chapter{13.01.2020, \textit{Dystrybucje - własności, delta Diraca}}
\subfile{tex/w24.tex}
\chapter{17.01.2020, \textit{Wzór Greena, $\Delta \frac{1}{r} = \delta$ }}
\subfile{tex/w25.tex}

\end{document}
