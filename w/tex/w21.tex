\documentclass[../main.tex]{subfiles}
\graphicspath{
    {"../img/"}
    {"img/"}
}

\begin{document}
Do pytania o $L_1$ i $L_2$.
\begin{align*}
    \int\limits_{-\infty}^{+\infty} |f| &= \int\limits_{0}^1 (x)^{-\frac{2}{3}} = 3\\
    \int\limits_{-\infty}^{+\infty} |f|^2 &= \int\limits_{0}^{1} (x)^{-\frac{4}{3}} \text{ nie istnieje}\\
    \int\limits_{-\infty}^{+\infty} |g| &=\int\limits_1^{+\infty} (x)^{-\frac{2}{3}} \text{ nie istnieje}\\
    \int\limits_{-\infty}^{+\infty} |g|^2 &= \int\limits_{1}^{+\infty} (x)^{-\frac{2}{3}} = 3\\
.\end{align*}
Czyli $f$ - klasy $L_1$, $g$ - klasy $L_2$
\subsection{Własności (transformaty Fouriera)}
\begin{enumerate}
    \item Niech $\alpha, \beta\in\mathbb{R}$. $f,g$ - klasy $L_1$, wówczas
        \[
            \mathcal{F}(\alpha f + \beta g) = \alpha \mathcal{F} f + \beta \mathcal{F} g
        .\]
    (z liniowości całki)
\item Niech $f,g$ - klasy $L_1$, wówczas
    \[
        \int_{-\infty}^{+\infty}f(x) \hat{g}(x) dx = \int\limits_{-\infty}^{+\infty}\hat{f}(x)g(x)dx
    .\]
\end{enumerate}
\begin{proof}
    (z twierdzenia Foubiniego)
\[
    \hat{g}(x) = \int\limits_{-\infty}^{+\infty}g(k) e^{-2\pi i kx}dk
.\]
\[
    \int\limits_{-\infty}^{+\infty}f(x)dx \int\limits_{-\infty}^{+\infty}g(k)e^{-2\pi i kx}dk = \int\limits_{-\infty}^{+\infty}g(k)dk \int\limits_{-\infty}^{+\infty}f(x) e^{-2\pi i kx}dx = \int\limits_{-\infty}^{+\infty}g(k)\hat{f}(k)dk
.\]
\end{proof}
\textbf{Obserwacja:} chcemy rozwiązać równanie:
\[
    \left( f(t) \right)'' + \omega^2 f(t) = g(t)
.\]
Załóżmy, że nasz $f$ :
        \[ f(t) = \int\limits_{-\infty}^{+\infty} h(k) e^{-2\pi i kt}dk
        .\]
        Dajmy na to, że
\[
    g(t) = \int\limits_{-\infty}^{+\infty}\omega(k)e^{-2\pi ik t}dk
.\]
\[
    f'(t) = -2\pi ik \int\limits_{-\infty}^{+\infty}h(k)e^{-2\pi i k t}dk
.\]
\[
    f''(t) = (-2\pi i k)^2 \int\limits_{-\infty}^{+\infty}h(k)e^{-2\pi i k t}dk
.\]
Po podstawieniu do oscylatora, uzyskujemy napis
\[
    \int\limits_{-\infty}^{+\infty}\left[(-2\pi i k)^2 h(k) + \omega^2 h(k) - w(k)\right]e^{-2\pi i k t}dk = 0
,\]
co by oznaczało tyle, że
\[
    \left(-4 \pi^2 k^2 + \omega^2\right)h(k) = w(k)
.\]
Czyli
\[
    h(k) = \frac{w(k)}{-4\pi^2 k^2 + \omega^2}
.\]
Ale wiemy, że
\[
    f(t) = \int\limits_{-\infty}^{+\infty}\frac{w(k)}{\omega^2 - 4 \pi^2 k^2}e^{-2\pi i kt}dt
.\]
\textbf{Obserwacja: }Jeżeli $f$ - klasy $L_1$ i $f'$ - klasy $L_1$, to $\mathcal{F}\left(f'\right)(x) = 2\pi i x (\mathcal{F}f)$
\[
    \int\limits_{-\infty}^{+\infty}f'(k)e^{-2\pi i kx}dk = \left. f(k)e^{-2 \pi i kx} \right|_{-\infty}^{+\infty} - \left( -2\pi i x \right) \underbrace{\int\limits_{-\infty}^{+\infty}f(k)e^{-2\pi i kx}dk}_{\mathcal{F}(f)}
.\]
Zauważmy, że jeżeli $\int_{-\infty}^{+\infty}|f| < +\infty$, to znaczy, że $\lim_{x \to \infty}f(x) = 0$ i $\lim_{x \to -\infty}f(x) = 0$, oraz skoro $f'$ - klasy $L_1$, to
\[
    f(x) - f(0) = \int\limits_{0}^x f'(k)dk
.\]
Skoro $f'$ - klasy $L_1$, to znaczy, że
\[
    \lim_{x \to \infty}\left| f(x) - f(0) \right| \le \left| \int\limits_{0}^{+\infty}f'(k)dk \right| \le M
.\]
Widzimy zatem, że $\lim_{x \to +\infty}\left| f(x) \right| \le M$ znaczy, że jeżeli $\int_{-\infty}^{+\infty}|f(x)| < +\infty$, to znaczy, że
\[\lim_{x \to \infty}f(x) = 0\quad \Box\]
Zatem
\[
    \mathcal{F}\left( f' \right) (x) = (2\pi i x)(\mathcal{F}f)(x)
,\]
(jeżeli $f,f',\ldots,f^{(m)}$ - klasy $L_1$ ) i ogólniej
\[
    \mathcal{F}(f^{(m)}(x)) = \left( 2\pi i x \right) ^m (\mathcal{F}f)(x)
.\]
\textbf{Obserwacja: }Niech $f$ - klasy $L_1$, wówczas $\frac{d}{dx} \left( \mathcal{F}f \right) (x) = -2\pi i \mathcal{F}(xf)$
\begin{proof}
    \[
        \frac{d}{dx}\left( \mathcal{F}f \right) (x) = \lim_{h \to 0}\frac{(\mathcal{F}f)(x+h) - (\mathcal{F}f)(x)}{h} = \lim_{h \to 0}\frac{1}{h}\int\limits_{-\infty}^{+\infty}\left( f(k)e^{-2\pi i k(x+h)} - f(k)e^{-2\pi i kx} \right)dk =
    \]
    \begin{equation}
        \label{eqn:eq21-1}
        = \lim_{h \to 0}\int\limits_{-\infty}^{+\infty}f(k)e^{-2\pi i k x}\left( \frac{e^{-2\pi i k h} - 1}{h} \right) dt\tag{$\star$}
    \end{equation}
Ale
\[
    \lim_{h \to 0}\frac{e^{-2\pi i k h} - 1}{h} \overset{\text{H}}{=} \lim_{h \to 0} \frac{-2\pi i ke^{-2\pi i kh}}{1} = -2\pi i k
.\]
Zatem dalej mamy
\[
    \eqref{eqn:eq21-1} = \int\limits_{-\infty}^{+\infty}-2\pi i k f(k)e^{-2\pi i kx}dk = -2\pi i \widehat{\left( xf \right) }
.\]
\end{proof}
\subsection{Transformata odwrotna}
\begin{enumerate}
    \item policzmy $\int_{-\infty}^{+\infty}\left( \mathcal{F}f \right) (x)dx$
        \[
            I = \int\limits_{-\infty}^{+\infty}dx\int\limits_{-\infty}^{+\infty}f(k)e^{-2\pi i k x}dk
        .\]
    Wcześniej napisaliśmy $\int f \hat{g} = \int \hat{f} g$. No to weźmy $\int\limits_{-\infty}^{+\infty} 1 \cdot e^{-2\pi i kx}dk$, ale to jeszcze nie teraz, bo taka całka jeszcze nie istnieje. Zauważmy, że
        \[
            I = \lim_{\varepsilon \to 0}\int\limits_{-\infty}^{+\infty}dx \int\limits_{-\infty}^{+\infty}f(k)e^{-2\pi i k x}e^{-\varepsilon|x|}dk =
        \]
    \[
        \lim_{\varepsilon \to 0^+} \int\limits_{-\infty}^{+\infty}f(k)\int\limits_{-\infty}^{+\infty}e^{-2\pi i kx}e^{-\varepsilon|x|}dx
    .\]
Policzmy
\[
    \int\limits_{-\infty}^{+\infty}e^{-2\pi i kx}e^{-\varepsilon|x|}dx = \int\limits_{0}^{+\infty}e^{-2\pi i kx}e^{-\varepsilon|x|}dx + \int\limits_{-\infty}^{0}e^{-2\pi i kx}e^{\varepsilon|x|}dx =
\]
\begin{equation}
    \label{eqn:eq21-2}
    \int\limits_{0}^{+\infty}e^{(-2\pi i k - \varepsilon)x}dx + \int\limits_{-\infty}^{0}e^{(-2\pi ik + \varepsilon)x}dx = \frac{1}{-2\pi i k -\varepsilon}\left.e^{(-2\pi i k - \varepsilon)x}\right|_{0}^{+\infty} + \frac{1}{-2\pi i k + \varepsilon}\left.e^{-2\pi ik + \varepsilon}\right|_{-\infty}^0 \tag{$\star\star$}
\end{equation}
Ale $e^{-2\pi i kx} \cdot e^{-\varepsilon x}\underset{x\to +\infty}{\longrightarrow} 0$
\[
    \eqref{eqn:eq21-2} = \frac{-1}{-2\pi i k - \varepsilon} + \frac{1}{-2 \pi i k + \varepsilon} = \frac{1}{\varepsilon + 2 \pi i k} + \frac{1}{\varepsilon - 2\pi i k} = \frac{2 \varepsilon}{\varepsilon^2 + (2\pi k)^2}
.\]
Zatem
\[
    I = \lim_{\varepsilon \to 0^+} \int\limits_{-\infty}^{+\infty}f(k) \frac{2\varepsilon}{\varepsilon^2 + (2\pi k)^2}dk
.\]
\[
    I = \lim_{\varepsilon \to 0} \frac{1}{2\pi}\int\limits_{-\infty}^{+\infty}f\left( \frac{\varepsilon L}{2\pi} \right) \frac{2 \varepsilon}{\varepsilon^2 + (\varepsilon L)^2} \cdot \varepsilon \cdot dL = \lim_{\varepsilon \to 0}\frac{1}{2\pi}\int\limits_{-\infty}^{+\infty}f\left( \frac{\varepsilon L}{2\pi} \right) \frac{2 \varepsilon^2 dL}{\varepsilon^2(1+\varepsilon)}
.\]
\[
    I = \lim_{\varepsilon \to 0}\frac{2}{2\pi }\int\limits_{-\infty}^{+\infty}f\left( \frac{2\varepsilon L}{2\pi } \right) \frac{1}{1 + L^2}dL = \frac{2}{2\pi}\int\limits_{-\infty}^{+\infty}f(0)\frac{dL}{1+L^2} = \frac{2 f(0)}{2\pi }\left( arctg(L)\right)_{-\infty}^{+\infty} = \frac{f(0)}{\pi} \left[ \frac{\pi}{2} + \frac{\pi}{2} \right] = \frac{f(0)}{\pi} \cdot \pi = f(0)\quad \Box
.\]
Niech $f_L(x) = f(x+L)$. Wtedy
 \[
     \mathcal{F}\left( f_L(x) \right) = \int\limits_{-\infty}^{+\infty}f_L(k) e^{-2\pi i k x}dk = \int\limits_{-\infty}^{+\infty}f(L+k)e^{-2\pi ikx}dk =
\]
\[
    = \int\limits_{-\infty}^{+\infty}f(k') e^{-2\pi i x(k' - L)}dk' = \int\limits_{-\infty}^{+\infty}e^{2\pi ix L}f(k')e^{-2\pi ixk'}dk' = e^{2\pi ixL}\left( \mathcal{F}f \right)
.\]
Policzmy całkę $\int_{-\infty}^{+\infty}(\mathcal{F}f_L)(x)$. Wiemy, że
\[
    f_L(0) = \int\limits_{-\infty}^{+\infty}\mathcal{F}f_L = \int\limits_{-\infty}^{+\infty}e^{2\pi i L}\mathcal{F}f
.\]
Czyli
\[
    f(L) = \int\limits_{-\infty}^{+\infty}\hat{f}e^{2\pi i L}dL
.\]
\end{enumerate}
Mamy wzór na transformatę odwrotną, czyli wiemy, że jeżeli $\hat{f}(x) = \int_{-\infty}^{+\infty}f(k)e^{-2\pi i kx}dk$, to $f(x) = \int_{-\infty}^{+\infty}\hat{f}(k) e^{2\pi i kx}dk$
\end{document}
